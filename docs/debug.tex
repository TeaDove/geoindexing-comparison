\documentclass{article}
\usepackage{graphicx} % Required for inserting images
\usepackage[english, russian]{babel}


\title{Сравнительный анализ применения методов геопоиска и структур геоиндекции в высоконагруженных системах}
\author{Петер Ибрагимов}
\date{May 2023}

\begin{document}
\maketitle


\section{Список сокращений}
\begin{itemize}
    \item СУБД - система управления базами данных
    \item БД - базами данных

\end{itemize}
\section{Теоретический аспект геопоиска и структур геоиндексации}
\subsection{Задача геоиндекации}

Геоиндекацией называются процессы поиска по данным, ключи которых представляют кортеж координат. Если решать задачу относительно декардовой плоскости, то ключом будем пара (X, Y).  В данной работе основной акцент будет ставиться на задаче поиска для земного шара по паре (X, Y), где X и Y - долгота и широта. Указанную пару можно также названать геокоординатами.\\

Основные задачи поиска можно сформулировать следующим образом:
\begin{enumerate}
    \item Найти все объекты Х, находящиеся на растояние У от объекта Z. Эту задачу можно переформулировать следующим образом: найти все объекты множества Х в окружности c радиусом Y и центром в Z.
    \item Найти X ближайщих объектов к объекту Z. Эта задача также известна как задача поиска ближайших соседей.
    \item Найти объекты Х, входящие в полигон Z. Частный случай данной задачи является задачей 1.
    \item Найти пересечение полигонов X и Y.
\end{enumerate}



\end{document}
