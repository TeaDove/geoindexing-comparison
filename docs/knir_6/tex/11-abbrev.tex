\Abbreviations %% Список обозначений и сокращений в тексте
В настоящей КНИР применяют следующие сокращения и обозначения:

\noindent СУБД --- Система управления базами данных

\noindent БД --- База данных

\noindent SQL (Structured Query Language) --- Язык структурированных запросов. Декларативный язык запросов для взаимодействия с СУБД. Используется в таких базах данных как: Postgres, MySQL, MSSQL, SQLite и тд.

\noindent RPS (Requests per second) --- Запросов за секунду. Метрика оценки нагруженности системы по количество запросов от клиентов серверу в секундах.

\noindent KNN (k-nearest neighbors) - К-ближайщих соседей, задача поиска К-ближайших соседей к точке А по метрике D.

%%% Local Variables:
%%% mode: latex
%%% TeX-master: "rpz"
%%% Stop:
