% % Список литературы при помощи BibTeX
% Юзать так:
%
% pdflatex rpz
% bibtex rpz
% pdflatex rpz

% \bibliographystyle{gost2800}
\bibliography{rpz}

\begin{thebibliography}{9}

\bibitem{nnq} %% 1 
Roussopoulos N., Kelley S., Vincent F. Nearest Neighbor Queries // Proceedings ACM SIGMOD. --- 1995. --- V.24. --- №2. --- P. 71-79.

\bibitem{guttmanRtree} %% 2
Guttman A. R-trees: A Dynamic Index Structure for Spatial Searching // Proceedings of ACM SIGMOD. --- 1984. --- V.14. --- №2. --- P. 47-57.

\bibitem{beckmannRStarTree} %% 3
Beckmann N., Kriegel H.P., Schneider R. [et al.]. The R*-tree: An Efficient and Robust Access Method for Points and Rectangles // Proceedings of ACM SIGMOD. --- 1990. --- V.19. --- №2. --- P. 322–331.

\bibitem{sunRStartree} %% 4
Sun L., Jin B. Improving NoSQL Spatial-Query Processing with Server-Side In-Memory R*-Tree Indexes for Spatial Vector Data // Sustainability. --- 2023. --- V.15, --- №1. --- P. 1-23. 

\bibitem{fedorovskieRTree} %% 5
Федотовский П. В. и др. Сортировать или нет: экспериментальное сравнение R-Tree и B+-Tree в транзакционной системе для упорядоченной выдачи // Труды Института системного программирования РАН. --- 2014. --- Т.26. --- №.4. --- С.73-90.


\bibitem{liuGBTree} %% 6
Liu Q., Tan X., Huang F., [et al.]. GB-Tree: An efficient LBS location data indexing method // 2014 The Third International Conference on Agro-Geoinformatics, Beijing, China. --- 2014. --- №9. --- P. 1-5.

\bibitem{gantiMPTrie} %% 7
Ganti R., Srivatsa M., Agrawal D., [et al.]. MP-trie: fast spatial queries on moving objects // Proceedings of the Industrial Track of the 17th International Middleware Conference. --- 2016. --- №12. --- P. 1-6.

\bibitem{guanSTHash} %% 8
Guan X., Bo C., Li Z., [et al.]. ST-hash: An efficient spatiotemporal index for massive trajectory data in a NoSQL database //  2017 25th International Conference on Geoinformatics, Buffalo, NY, USA. --- 2017. --- №8 --- P. 1-7.

\bibitem{jiajunGeohash} %% 9
Liu J., Li H., Gao Y., [et al.]. A geohash-based index for spatial data management in distributed memory // IEEE 2014 22nd International Conference on Geoinformatics. --- 2014. --- №6. --- P. 1-4.

\bibitem{sahrIndexingSystems} %% 10
Sahr K. Central Place Indexing: Hierarchical Linear Indexing Systems for Mixed-Aperture Hexagonal Discrete Global Grid Systems // Cartographica: The International Journal for Geographic Information and Geovisualization. --- 2019. --- V.54. --- №3. --- P.16-29.

\bibitem{sidorovGeohash} %% 11
Сидоров, И. Ю., Армяков, А. О., Байтин, А. А., [и др.]. Хранилище точечных геообъектов // Минцевские чтения. --- 2015. --- C. 234-247.

\bibitem{balkicGeohash} %% 12
Balkić Z., Šoštarić D., Horvat G. GeoHash and UUID identifier for multi-agent systems // Proceedings of the 6th KES International Conference on Agent and Multi-Agent Systems: Technologies and Applications. --- 2012. --- №6. --- P. 290–298.

\bibitem{bohuiGeohashH2S2} %% 13
Bohui, J., Weifeng, Z. Comparative Analysis of GeoHash, Google S2 and Uber H3 as Global Geographic Grid Coding Methods // Geography & Geographic Information Science. --- 2024. --- V.40. --- №2. --- P. 19

\bibitem{membreyMongodb} %% 14
Membrey P., Plugge E., Hawkins T. The definitive guide to MongoDB: the
noSQL database for cloud and desktop computing.--- New York.: Apress, 2010.--- P. 328.

\bibitem{heHBase} %% 16
S. He S., Chu L., Li X. Spatial query processing for location based application on Hbase // 2017 IEEE 2nd International Conference on Big Data Analysis (ICBDA), Beijing, China. --- 2017. --- №3. --- P. 110-114.

\bibitem{bayerIndices} %% 15
Bayer R., McCreight E. Organization and maintenance of large ordered indices // Proceedings of the 1970 ACM SIGFIDET (Now SIGMOD) Workshop on Data Description Access and Control. --- 1970. --- №11. --- P. 107–141.

\bibitem{comerBTree} %% 17
Comer, D. Ubiquitous B-tree // ACM Computing Surveys. --- 1979. --- V.11 --- №2. --- P. 121-137.

\bibitem{gulakovStructured} %% 18
Гулаков В.К. Многомерные структуры данных.--- Брянск.: БГТУ, 2010.--- 387 c.

\bibitem{redisGeo} %% 19
Muradova G., Hematyar M., Jamalova J. Advantages of Redis in-memory database to efficiently search for healthcare medical supplies using geospatial data // 2022 IEEE 16th International Conference on Application of Information and Communication Technologies. --- 2022. --- №10. --- P. 1-5.

\bibitem{ibragimov} %% 20
Ибрагимов П.И. Сравнительный анализ алгоритмов геопоиска и пространственных индексов в высоконагруженных системах // 79-е дни науки студентов МИСиС: международные, межвузовские и институтские научно-технические конференции. Тезисы докладов. — М.: МИСиС, 2022. 

\end{thebibliography}

%%% Local Variables: 
%%% mode: latex
%%% TeX-master: "rpz"
%%% End: 
