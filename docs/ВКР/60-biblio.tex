% % Список литературы при помощи BibTeX
% Юзать так:
%
% pdflatex rpz
% bibtex rpz
% pdflatex rpz

% \bibliographystyle{gost2800}
\bibliography{rpz}

\begin{thebibliography}{9}

\bibitem{nnq} %% 1 
Roussopoulos N., Kelley S., Vincent F. Nearest Neighbor Queries // Proceedings ACM SIGMOD. --- 1995. --- V.24. --- №2. --- P. 71-79.

\bibitem{guttmanRtree} %% 2
Guttman A. R-trees: A Dynamic Index Structure for Spatial Searching // Proceedings of ACM SIGMOD. --- 1984. --- V.14. --- №2. --- P. 47-57.

\bibitem{beckmannRStarTree} %% 3
Beckmann N., Kriegel H.P., Schneider R. [et al.]. The R*-tree: An Efficient and Robust Access Method for Points and Rectangles // Proceedings of ACM SIGMOD. --- 1990. --- V.19. --- №2. --- P. 322–331.

\bibitem{sunRStartree} %% 4
Sun L., Jin B. Improving NoSQL Spatial-Query Processing with Server-Side In-Memory R*-Tree Indexes for Spatial Vector Data // Sustainability. --- 2023. --- V.15, --- №1. --- P. 1-23. 

\bibitem{fedorovskieRTree} %% 5
Федотовский П. В. и др. Сортировать или нет: экспериментальное сравнение R-Tree и B+-Tree в транзакционной системе для упорядоченной выдачи // Труды Института системного программирования РАН. --- 2014. --- Т.26. --- №.4. --- С.73-90.


\bibitem{liuGBTree} %% 6
Liu Q., Tan X., Huang F., [et al.]. GB-Tree: An efficient LBS location data indexing method // 2014 The Third International Conference on Agro-Geoinformatics, Beijing, China. --- 2014. --- №9. --- P. 1-5.

\bibitem{gantiMPTrie} %% 7
Ganti R., Srivatsa M., Agrawal D., [et al.]. MP-trie: fast spatial queries on moving objects // Proceedings of the Industrial Track of the 17th International Middleware Conference. --- 2016. --- №12. --- P. 1-6.

\bibitem{guanSTHash} %% 8
Guan X., Bo C., Li Z., [et al.]. ST-hash: An efficient spatiotemporal index for massive trajectory data in a NoSQL database //  2017 25th International Conference on Geoinformatics, Buffalo, NY, USA. --- 2017. --- №8 --- P. 1-7.

\bibitem{jiajunGeohash} %% 9
Liu J., Li H., Gao Y., [et al.]. A geohash-based index for spatial data management in distributed memory // IEEE 2014 22nd International Conference on Geoinformatics. --- 2014. --- №6. --- P. 1-4.

\bibitem{sahrIndexingSystems} %% 10
Sahr K. Central Place Indexing: Hierarchical Linear Indexing Systems for Mixed-Aperture Hexagonal Discrete Global Grid Systems // Cartographica: The International Journal for Geographic Information and Geovisualization. --- 2019. --- V.54. --- №3. --- P.16-29.

\bibitem{sidorovGeohash} %% 11
Сидоров, И. Ю., Армяков, А. О., Байтин, А. А., [и др.]. Хранилище точечных геообъектов // Минцевские чтения. --- 2015. --- C. 234-247.

\bibitem{balkicGeohash} %% 12
Balkić Z., Šoštarić D., Horvat G. GeoHash and UUID identifier for multi-agent systems // Proceedings of the 6th KES International Conference on Agent and Multi-Agent Systems: Technologies and Applications. --- 2012. --- №6. --- P. 290–298.

\bibitem{bohuiGeohashH2S2} %% 13
Bohui, J., Weifeng, Z. Comparative Analysis of GeoHash, Google S2 and Uber H3 as Global Geographic Grid Coding Methods // Geography & Geographic Information Science. --- 2024. --- V.40. --- №2. --- P. 19

\bibitem{membreyMongodb} %% 14
Membrey P., Plugge E., Hawkins T. The definitive guide to MongoDB: the
noSQL database for cloud and desktop computing.--- New York.: Apress, 2010.--- P. 328.

\bibitem{heHBase} %% 16
S. He S., Chu L., Li X. Spatial query processing for location based application on Hbase // 2017 IEEE 2nd International Conference on Big Data Analysis (ICBDA), Beijing, China. --- 2017. --- №3. --- P. 110-114.

\bibitem{bayerIndices} %% 15
Bayer R., McCreight E. Organization and maintenance of large ordered indices // Proceedings of the 1970 ACM SIGFIDET (Now SIGMOD) Workshop on Data Description Access and Control. --- 1970. --- №11. --- P. 107–141.

\bibitem{comerBTree} %% 17
Comer, D. Ubiquitous B-tree // ACM Computing Surveys. --- 1979. --- V.11 --- №2. --- P. 121-137.

\bibitem{gulakovStructured} %% 18
Гулаков В.К. Многомерные структуры данных.--- Брянск.: БГТУ, 2010.--- 387 c.

\bibitem{redisGeo} %% 19
Muradova G., Hematyar M., Jamalova J. Advantages of Redis in-memory database to efficiently search for healthcare medical supplies using geospatial data // 2022 IEEE 16th International Conference on Application of Information and Communication Technologies. --- 2022. --- №10. --- P. 1-5.

\bibitem{ibragimov} %% 20
Ибрагимов П.И. Сравнительный анализ алгоритмов геопоиска и пространственных индексов в высоконагруженных системах // 79-е дни науки студентов МИСиС: международные, межвузовские и институтские научно-технические конференции. Тезисы докладов. — М.: МИСиС, 2022. 

\end{thebibliography}

% \begin{enumerate}[1.]
%     \item N. Roussopoulos, S. Kelley and F. Vincent. Nearest Neighbor Queries // Proceedings ACM SIGMOD. --- 1995. ---T.24. ---№2. ---P. 71-79.
%     \item A. Guttman. R-trees: A Dynamic Index Structure for Spatial Searching // Proceedings of ACM SIGMOD. ---1984. ---T.14. ---№2. --- P. 47-57.
%     \item N. Beckmann, H.P. Kriegel, R. Schneider and B. Seeger. The R*-tree: An Efficient and Robust Access Method for Points and Rectangles // Proceedings of ACM SIGMOD. ---1990. ---T.19. ---№2. ---P. 322–331.
%     \item K. Sahr. Central Place Indexing: Hierarchical Linear Indexing Systems for Mixed-Aperture Hexagonal Discrete Global Grid Systems // Cartographica: The International Journal for Geographic Information and Geovisualization, ---2019. ---T.54. ---№3. ---P.16-29.
%     \item Федотовский П. В. и др. Сортировать или нет: экспериментальное сравнение R-Tree и B+-Tree в транзакционной системе для упорядоченной выдачи //Труды Института системного программирования РАН. – 2014. – Т. 26. – №. 4. – С. 73-90.
%     \item  Jiajun Liu, Haoran Li, Yong Gao, Hao Yu, Dan Jiang. [IEEE 2014 22nd International Conference on Geoinformatics - Kaohsiung, Taiwan (2014.6.25-2014.6.27)] 2014 22nd International Conference on Geoinformatics - A geohash-based index for spatial data management in distributed memory. 2014
%     \item Q. Liu, X. Tan, F. Huang, C. Peng, Y. Yao and M. Gao, "GB-Tree: An efficient LBS location data indexing method," 2014 The Third International Conference on Agro-Geoinformatics, Beijing, China, 2014, pp. 1-5, doi: 10.1109/Agro-Geoinformatics.2014.6910659. https://ieeexplore.ieee.org/document/6910659
%     \item S. He, L. Chu and X. Li, "Spatial query processing for location based application on Hbase," 2017 IEEE 2nd International Conference on Big Data Analysis (ICBDA), Beijing, China, 2017, pp. 110-114, doi: 10.1109/ICBDA.2017.8078787. https://ieeexplore.ieee.org/abstract/document/8078787
%     \item  Beckmann, N., Kriegel, H.-P., Schneider, R., Seeger, B.: The R*-tree: an efficient and robust access method for points and rectangles. In: Proceedings of the 1990 ACM SIGMOD International Conference on Management of Data, pp. 322–331 (1990). https://dl.acm.org/doi/abs/10.1145/93597.98741
%     \item R. Bayer and E. McCreight, "Organization and maintenance of large ordered indices", Proceedings of the 1970 ACM SIGFIDET (Now SIGMOD) Workshop on Data Description Access and Control, 1970. https://ieeexplore.ieee.org/document/10322993
%     \item Ganti, R.K., Srivatsa, M., Agrawal, D., Zerfos, P., Ortiz, J.: MP-trie: fast spatial queries on moving objects. In: Proceedings of the Industrial Track of the 17th International Middleware Conference, p. 1. ACM (2016) https://dl.acm.org/doi/abs/10.1145/3007646.3007653
%     \item Sun, L.; Jin, B. Improving NoSQL Spatial-Query Processing with Server-Side In-Memory R*-Tree Indexes for Spatial Vector Data. Sustainability 2023, 15, 2442.
%     \item Ибрагимов П.И. Сравнительный анализ алгоритмов геопоиска и пространственных индексов в высоконагруженных системах // 79-е дни науки студентов МИСиС: международные, межвузовские и институтские научно-технические конференции. Тезисы докладов. — М.: МИСиС, 2022.
%     \item X. Guan, C. Bo, Z. Li and Y. Yu, "ST-hash: An efficient spatiotemporal index for massive trajectory data in a NoSQL database," 2017 25th International Conference on Geoinformatics, Buffalo, NY, USA, 2017, pp. 1-7, doi: 10.1109/GEOINFORMATICS.2017.8090927. keywords: {Spatiotemporal phenomena;Trajectory;Indexes;Erbium;Binary codes;ST-Hash;Spatiotemporal index;Spatiotemporal range query;Trajectory data;NoSQL},
%     \item Membrey P., Plugge E., Hawkins T. The definitive guide to MongoDB: the
% noSQL database for cloud and desktop computing. Apress, 2010
%     \item Сидоров, И. Ю., Армяков, А. О., Байтин, А. А., & Гунин, А. В. (2015). Хранилище точечных геообъектов. In Минцевские чтения (pp. 234-247).
%     \item Гулаков В.К. Многомерные структуры данных. Брянск, БГТУ, 2010, c. 387.
%     \item Balkić Z., Šoštarić D., Horvat G. GeoHash and UUID identifier for multi-agent
% systems. Agent and Multi-Agent Systems. Technologies and Applications. Berlin
% Springer, Heidelberg, 2012, pp. 290–298
%     \item Bohui, J. I. A. N. G., & Weifeng, Z. H. O. U. (2024). Comparative Analysis of GeoHash, Google S2 and Uber H3 as Global Geographic Grid Coding Methods. Geography & Geographic Information Science, 40(2).
%     \item Comer, D. (1979). Ubiquitous B-tree. ACM Computing Surveys (CSUR), 11(2), 121-137.
    
% \end{enumerate}

%%% Local Variables: 
%%% mode: latex
%%% TeX-master: "rpz"
%%% End: 
