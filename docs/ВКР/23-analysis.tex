\section{1.4 Сравнительный анализ}

Чтобы дать корректный ответ на вопрос о том, какую структуру данных стоит использовать, и какие алгоритмы к ней применять, стоит добавить следующие ограничения:
\par а) не будет учитываться время записи на диск. Предполагается, что все операции производятся при использовании оперативной памяти. Данное ограничение обусловлено тем, что операции сериализации и десериализации индексов на постоянное запоминающие устройство в общем случае идентичны для разных индексов;
\par б) требуется учитывать как теоретические показатели, то есть O-нотацию, так и практические результаты.


\subsection{1.4.1 Качественный сравнительный анализ}
Методы геопоиска и пространственных индексов могут использоваться в различных сценариях, включая навигационные приложения, сервисы доставки еды и товаров, системы мониторинга транспорта и т.д.\cite{heHBase}

Как в задачах анализа данных, так и при разработке операционных сервисов, лидирует индекс R-Tree. Скорее всего это связано со следующими причинами:
\par а) используется в популярном аддоне PostGIS для Postgres;
\par б) эффективен;
\par б) имеет множество методов помимо BBox, KNN и тд.

Таким образом, в общем случае перед разработчиком и не встает вопрос о выборе индекса. Но, важно отметить, что есть множество ситуаций, из-за которых выбор может пасть на другие методы и индексы, например, требования по кластеризации. 

Выбор методов геопоиска и пространственных индексов зависит от конкретной задачи и сценария использования. Необходимо учитывать требования к производительности, точности и эффективности поиска объектов на карте. 

%%% Local Variables:
%%% mode: latex
%%% TeX-master: "rpz"
%%% End:
