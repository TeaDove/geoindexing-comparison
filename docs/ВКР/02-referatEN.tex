\ReferatEN
%%Полный перевод текста РЕФЕРАТА на английский язык.

FQW 62 p., 24 fig., 4 f-l, 1 tabl., 20 sources, 2 app.

\noindent TREE-LIKE DATA STRUCTURES, ABSTRACT DATA TYPES, INDEXES, SPACIAL INDEXES, R-TREE, KD-TREE, GEOHASH, H3, B-TREE, GEOSEARCH

%%Ключевые слова от 5 до 15 слов или словосочетаний КАПСОМ. ТОЧКУ в конце  НЕ СТАВИТЬ 

Object of research: spacial indexes and search algorithms on geo data.

The purpose of the work: develop software for practical analysis of popular geo algorithms and spacial indexes, which could potentially be used in high loaded systems. Mathematical (theoretical) analysis of the reported methods and algorithms. Modification of the existing algorithms, development of new algorithms. 

Methods of work: discrete mathematics, graph theory, performance analysis of indexes.

The result of the work: loadtesting system for indexes. New indexes, optimizations of existing indexes. Performance scoring of indexes.

Scope of the results: backend development, DBMS



%%% Local Variables:
%%% mode: latex
%%% TeX-master: "rpz"
%%% End:
