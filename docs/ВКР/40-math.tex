\chapter{МАТЕМАТИЧЕСКАЯ ПОСТАНОВКА}
\label{cha:math}

В работе будут проверять 3 основные операции над пространственными индексами, которые были подробно описаны в главе "Содержательная постановка задачи": Insert, KNN, RangeSearch.

Дано:
\begin{enumerate}
    \item $n \in \mathbb{N}$, $n$ - количество элементов в индексе
    \item $f(n) = x, x \in  \mathbb{Q}, x > 0$, где $f(n)$ - время выполнения операции в секундах
    \item $g(n) = y, y \in  \mathbb{Q}, y > 0$, где $g(n)$ - затраты оперативной памяти на операцию в байтах
\end{enumerate}
Для каждой выше указанной операции и для каждой тестируемой структуры требуется вычислить аксиоматическую сложность алгоритма:
\begin{align}
    f_{ax}(n) = O(f(n)) \\
    g_{ax}(n) = O(g(n))
\end{align}

Реализуемое программное обеспечение должно уметь находить $f(n)$ и $g(n)$ для заданных n и строить график зависимости $f(n)$ и $g(n)$ от n.
