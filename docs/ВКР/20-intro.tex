\myChapter{1 АНАЛИТИЧЕСКИЙ ОБЗОР ЛИТЕРАТУРЫ}
\label{cha:analysis}

В данной работе будут анализироваться только 3 задачи поиска по геоданным. Задачи можно сформулировать следующим образом:
Пусть есть множество точек, представляющих собой пару широта-долгота, отображающих некий объект на поверхности земли. 

Необходимо произвести операции:
\par а) BBox(bottomLeft Point, upperRight Point) Points - поиск всех объектов, которые входят в прямоугольник, определенный нижним левым углом (bottomLeft) и правым верхним углом (upperRight). Возвращается массив найденных точек;
\par б) KNN(p Point, k int) Points - поиск k ближайших объектов от точки p, возвращается массив найденных точек;
\par в) Insert(p Point) - вставка нового элемента (точки) в индекс. 

Важно отметить, что, зачастую, для прикладных задач также требуется решить задачу поиска в кругу: Найти все точки X, находящееся на заданном расстояние R от заданной точки Y. Данная задача не рассматривается в данной работе из-за того, что на части индексов она не может быть решена оптимально. Например, на индексе R-tree. При этом её все равно можно решить через задачу BBox: 
\par а) решить задачу BBox для квадрата, описывающего заданный круг;
\par б) простым перебором отсечь точки, что не входят в заданный круг.

Таким образом в данной работе данная задача не рассматривается. При этом задача BBox может быть также сведена к задаче поиска в кругу:
\par а) решить задачу поиска в кругу для круга, описывающего квадрат;
\par б) простым перебором отсечь точки, что не входят в заданный прямоугольник.

\section{1.1 Расстояние между двумя точками}
Указанные выше задачи требуют обусловить понятие расстояния между двумя точками на сфере земли. Для этого введем понятие геоточки: это точка, представляющая собой кортеж A(x, y), где x и y - широта и долгота соответственно. 
Расстоянием между двумя точками есть наименьшая прямая на плоскости земного шара, при этом ее вычисление может быть разным в зависимости от положенной задачи. \\
Поставим задачу. Даны геоточки $A(x_1, y_1), B(x_2, y_2)$. Требуется найти расстояние $d(A, B)$

\subsection{1.1.1 Евклидово расстояние}
В некоторых системах, например, расширение Postgis для СУБД Postgres для типа Geometry, используется евклидово расстояние
  \newline
\begin{equation} \label{eq}
\
\ d(A, B)=sqrt((x_1-x_2)^2+(y_1-y_2)^2).
\
\end{equation}
  \newline
Данный подход работает корректно только в части случаев, так, например, он выдаст правильный результат с точностью до 5 метров в случае поиска расстояния в пределах Москвы, но, расстояние в Африке будет отличаться в 2-3 раза относительно более строгих подходов.  
При этом важно отметить, что данная дистанция отвечает аксиомам метрики, поэтому ее корректно использовать при сравнении расстояний между двумя точками, например, при решении задаче KNN(К-ближайших соседей). 

\subsection{1.1.2 Расстояние гаверсинуса}

Расстояние гаверсинуса — это способ определения расстояния между двумя точками на поверхности Земли, учитывающий кривизну Земли. Оно используется в геопоиске для определения расстояния между заданной точкой и объектами в заданном радиусе.

Формула расстояния гаверсинуса выглядит следующим образом
    \newline
\begin{equation}\label{eq}
    \begin{aligned}
        d(A, B) = \\
        = 2R \arcsin\left(sqrt(\frac{1 - \cos\left(\Delta x \right) + \cos x_1 \cdot \cos x_2 \cdot \left(1 - \cos\left(\Delta y\right)\right)}{2})\right),
    \end{aligned}
\end{equation}
\par\vspace{1em}
\par\vspace{1em}
\noindent
где d - расстояние между двумя точками в километрах, R - радиус Земли (приблизительно 6371 км), $x_1$ и $x_2$ - широты двух точек в радианах, $y_1$ и $y_2$ - долготы двух точек в радианах.

Эта формула позволяет определить расстояние между точками с точностью до нескольких метров, что делает ее очень полезной для геопоиска. Однако она может быть достаточно ресурсоёмкой при работе с большими объемами геоданных, что делает ее медленнее по сравнению с Евклидовым расстоянием. 

\subsection{1.1.3 Геодезическое расстояние}
Геодезическое расстояние — это расстояние между двумя точками на поверхности Земли, измеренное вдоль кратчайшей линии (геодезической линии) между этими точками. Геодезическая линия — это кривая на поверхности Земли, которая имеет наименьшую длину между двумя точками.

Геодезическое расстояние учитывает кривизну Земли и может отличаться от расстояния гаверсинуса, особенно на больших расстояниях и при использовании разных моделей формы Земли.

Для расчета геодезического расстояния используются различные методы, такие как метод Винсента, метод Гаусса-Крюгера и метод Хаверсина. Эти методы учитывают форму Земли и позволяют получить более точные результаты, чем простые формулы для расчета расстояния на плоскости.
    
Геодезическое расстояние широко используется в геопозиционировании, навигации, картографии и других областях, где требуется точное определение расстояния между двумя точками на поверхности Земли.

Формула вычисления геодезии
\begin{equation} \label{eq}
\ d(A, B) = R \cdot \arccos(\sin(x_1) \cdot \sin(x_1) + \cos(x_1) \cdot \cos(x_2) \cdot \cos(y_2 - y_1).
\end{equation}
\par
Таким образом, при выборе формулы поиска расстояния надо учитывать требования. Например, при разработке приложения кикшеринга, то есть аренды электронных самокатов, их можно комбинировать. На запросах, которые возвращают клиентам ближайшие самокаты от их текущей геопозиции по радиусу можно использовать метрику Евклида, так как ошибку даже на 20-30 метров можно нивелировать поиском по большему радиусу.  
При этом, в ситуации, когда требуется построить аналитику данных, лучше использовать расстояние геодезии, потому что в данных задачах крайне важна точность результата, а разница в скорости решение не так важна. 

