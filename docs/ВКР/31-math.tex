\section{2.2 Математическая постановка}
\label{cha:math}

В работе будут проверять 3 основные операции над пространственными индексами, которые были подробно описаны в главе «Содержательная постановка задачи»: Insert, KNN, BBox.

Дано:
\begin{enumerate}
    \item $n \in \mathbb{N}$, $n$ - количество элементов в индексе
    \item $f(n) = x, x \in  \mathbb{Q}, x > 0$, где $f(n)$ - время выполнения операции в микросекундах
\end{enumerate}
Для каждой выше указанной операции и для каждой тестируемой структуры требуется вычислить аксиоматическую сложность алгоритма:
  \newline
\begin{align}
    f_{ax}(n) = O(f(n)).
\end{align}
  \newline
Реализуемое программное обеспечение должно уметь находить $f(n)$ для заданных n и строить график зависимости $f(n)$ от n.
Разработанные и доработанные индексы должны корректно решать поставленые задачи. 
