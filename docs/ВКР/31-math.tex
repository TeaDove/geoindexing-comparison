\section{2.2 Математическая постановка}

В работе будут проверять 3 основные операции над пространственными индексами, которые были подробно описаны в главе «Содержательная постановка задачи»: Insert, KNN, BBox.

Дано:
\par а) $n \in \mathbb{N}$, $n$ - количество элементов в индексе;
\par б) $f(n) = x, x \in  \mathbb{Q}, x > 0$, где $f(n)$ - время выполнения операции в микросекундах.

Для каждой выше указанной операции и для каждой тестируемой структуры требуется вычислить аксиоматическую сложность алгоритма.
\vspace{1em}
\begin{align}
    f_{ax}(n) = O(f(n)).
\end{align}
\par\vspace{1em}

Реализуемое программное обеспечение должно уметь находить $f(n)$ для заданных n и строить график зависимости $f(n)$ от n.
Разработанные и доработанные индексы должны корректно решать поставленные задачи. 
