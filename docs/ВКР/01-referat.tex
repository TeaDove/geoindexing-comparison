\Referat %% Список обозначений и сокращений в тексте
%%Тут должен быть реферат по типу:

ВКР 59 с., 24 рис., 4 ф-л, 1 табл., 20 источн., 2 прил. %%это поменять под себя

\noindent ДРЕВОВИДНЫЕ СТРУКТУРЫ ДАННЫХ, АБСТРАКТНЫЕ СТРУКТУРЫ ДАННЫХ, ИНДЕКСЫ, ПРОСТРАНСТВЕННЫЕ ИНДЕКСЫ, R-TREE, KD-TREE, GEOHASH, H3, B-TREE, ГЕОПОИСК

%%Ключевые слова от 5 до 15 слов или словосочетаний КАПСОМ. ТОЧКУ в конце  НЕ СТАВИТЬ 

Объект исследования: пространственные индексы и алгоритмы поиска по геоданным. 

Цель работы: разработать программное обеспечение для практического анализа популярных алгоритмов геопоиска и пространственных индексов, которые потенциально могут использоваться в высоконагруженных системах, а также математический (теоретический) анализ указанных методов и алгоритмов. Доработка существующих и разработка новых алгоритмов под указанные задачи.

Методы проведения работы: дискретная математика, теория графов, анализ производительности индексов. 

Результат работы: система нагрузочного тестирования индексов. Новые индексы, оптимизация уже существующих индексов. Оценка качества индексов. 

Область применения результатов: бекенд разработка, СУБД.


%%Предметом исследования ВКР является: Цель работы..., В процессе работы проводились ..., В результате работы впервые были ...
%%Внимание!
%%Текст Реферата не более 1 страницы. Должен отражать:
%%- объект исследования;
%%- цель работы;
%%- методы проведения работы;
%%- результаты работы и их новизну;
%%- область применения результатов.+

%%% Local Variables:
%%% mode: latex
%%% TeX-master: "rpz"
%%% End:
