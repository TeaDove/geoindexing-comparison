\myChapter{2 СПЕЦИАЛЬНАЯ ЧАСТЬ}
\vspace{-2em}
\section{2.1 Содержательная постановка задачи}

Целью данной работы является разработка программного обеспечения для анализа пространственных индексов в высоконагруженных системах. Данное ПО должно уметь анализировать предложенные пространственные индексы: R-tree, R*-tree, KD-tree, Geohash+btree, H3+btree и другие.

Пусть G - это множество всех точек в индексе, и каждая точка обозначается символом $G_i$, где $i \in \mathbb{N}$, широта и долгота каждой точки обозначается как $X_i$ и $Y_i$, где $X_i \in \mathbb{R}$ и $Y_i \in \mathbb{R}$.

Анализ должен проводиться в разрезе трех операций:
\par а) Insert(p Point) - вставка новой точки p в индекс;
\par б) KNN(p Point, k int) Points -  поиск множества точек S размера k, находящихся на наименьшем расстоянии от точки p. $S=\{G_i | G_i \in G, dist(p, G_i) \leq dist(p, G_j),  \forall G_j \in G, G_j \notin S \}, |S| = k$;
\par в) BBox(bottomLeft Point, upperRight Point) Points --- поиск множества точек S, которых входят в прямоугольник, определяющийся левым нижним углом (bottomLeft) и правым верхним углом (upperRight). $S=\{G_i | G_i \in G, X_i \geq X_{bottomLeft}, Y_i \geq Y_{bottomLeft}, X_i \leq X_{upperRight}, Y_i \leq Y_{upperRight} )\}$.
Где $dist$ - функция расстояния между двумя точками. В данной работе применяется расстояние гаверсинуса. 

Важно отметить, что операция удаления точки не рассматривается из-за того, что, в основном, она не сильно отличается от операции вставки\cite{bayerIndices}, а также из-за того, что в высоконагруженных системах зачастую применяется подход soft-detele(от англ. мягкое удаление), в котором данные не удаляются, а лишь помечаются в СУБД как неактивные. Также, при проведении анализа данных работа происходит с read-only данными, то есть с теми данными, которые не изменяются. 

Для решения поставленных целей необходимо решить следующие задачи:
\par а) изучить существующие пространственные индексы;
\par б) разработать программное обеспечение для тестирования пространственных индексов;
\par в) реализовать в программном коде уже существующие индексы;
\par г) доработать существующие индексы и разработать новые индексы;
\par д) протестировать и отладить разработанное программное обеспечение;
\par е) проанализировать результаты работы программного обеспечения. Найти наиболее подходящие структуры под анализируемые задачи.
