\myChapter{2 СПЕЦИАЛЬНАЯ ЧАСТЬ}
\label{cha:statement}

\section{2.1 Содержательная постановка задачи}

Целью данной работы является разработка программного обеспечения для анализа пространственных индексов в высоконагруженных системах. Данное ПО должно уметь анализировать как минимум предложенные пространственные индексы: R-tree, R*-tree, KD-tree, Geohash+btree, H3+btree и другие.\\
Пусть G - это множество всех точек в индексе, и каждая точка обозначается символом $G_i$, где $i \in \mathbb{N}$, широта и долгота каждой точки обозначается как $X_i$ и $Y_i$, где $X_i \in \mathbb{R}$ и $Y_i \in \mathbb{R}$\\
Анализ должен проводится в разрезе трех операций:
\begin{enumerate}
    \item  Insert(p Point) - вставка нового точки p в индекс
    \item  KNN(p Point, k int) Points -  поиск множества точек S размера k, находящихся на наименьшем расстояние от точки p. $S=\{G_i | G_i \in G, dist(p, G_i) \leq dist(p, G_j),  \forall G_j \in G, G_j \notin S \}, |S| = k$
    \item  BBox(bottomLeft Point, upperRight Point) Points - поиск множества точек S, которых входят в прямоугольник, определяющийся левым нижним углом (bottomLeft) и правым верхнем углом (upperRight). $S=\{G_i | G_i \in G, X_i \geq X_{bottomLeft}, Y_i \geq Y_{bottomLeft}, X_i \leq X_{upperRight}, Y_i \leq Y_{upperRight} )\}$
\end{enumerate}
Где $dist$ - функция расстояния между двумя точками. В данном работе применяется расстояние гаверсинуса. 

Важно отметить, что операция удаления точки не рассматривается из-за того, что в основном она не сильно отличается от операции вставки, а также из-за того, что в высоко нагруженных системах, в основном, применяется подход soft-detele(от англ. мягкое удаление), в котором данные не удаляются, а лишь помечаются в СУБД как неактивные. Также, при проведении анализа данных, в основном, работа происходит с read-only данными, то есть теми данными, которые не меняются. 

Для решения поставленных целей необходимо решить следующие задачи:
\begin{enumerate}
    \item Изучить существующие пространственные индексы
    \item Разработать программное обеспечение для тестирования пространственных индексов.
    \item Реализовать в программном коде уже существующие индексы
    \item Доработать существующие индексы и разработать новые индексы. 
    \item Протестировать и отладить разработанное программное обеспечение.
    \item Проанализировать результаты работы программного обеспечения. Найти наиболее подходящие структуры под анализируемые задачи.
\end{enumerate}
