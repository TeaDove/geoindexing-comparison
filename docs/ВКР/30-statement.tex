\chapter{СОДЕРЖАТЕЛЬНАЯ ПОСТАНОВКА ЗАДАЧИ}
\label{cha:statement}

Целью данной работы является разработка программного обеспечения для анализа пространственных индексов в высоконагруженных системах. Данное ПО должно уметь анализировать как минимум предложенные пространственные индексы: R-tree, R*-tree, KD-tree, Geohash+trie, H3+trie.\\
Пусть G - это множество всех точек в индексе, и каждая точка обозначается символом $G_i$, где $G_i \in \mathbb{N}$.
Анализ должен проводится в разрезе трех операций:
\begin{enumerate}
    \item  Insert(p point) - вставка нового точки p в индекс
    \item  KNN(p point, k int) -  поиск множества точек S размера k, находящихся на наименьшем расстояние от точки p. $S=\{G_i | G_i \in G, dist(p, G_i) \leq dist(p, G_j),  \forall G_j \in G, G_j \neq G_i \}, |S| = k$
    \item  RangeSearch(p point, r float) - поиск множества точек S, которых входят в круг, определяющийся центром p и радиусом r. $S=\{G_i | G_i \in G, dist(p, G_i) \leq r)\}$
\end{enumerate}
Где $dist$ - функция расстояния между двумя точками. В данном работе применяется расстояние гаверсинуса. 

Важно отметить, что операция удаления точки не рассматривается из-за того, что в основном она не сильно отличается от операции вставки, а также из-за того, что в высоко нагруженных системах, в основном, применяется подход soft-detele(от англ. мягкое удаление), в котором данные не удаляются, а лишь помечаются в СУБД как неактивные. Также, при проведении анализа данных, в основном, работа происходит с read-only данными, то есть теми данными, которые не меняются. 

Для решения поставленных целей необходимо решить следующие задачи:
\begin{enumerate}
    \item Изучить существующие пространственные индексы
    \item Разработать программное обеспечение для тестирования пространственных индексов.
    \item Протестировать и отладить разработанное программное обеспечение.
    \item Проанализировать результаты работы программного обеспечения. Найти наиболее подходящие структуры под анализируемые задачи.
\end{enumerate}

