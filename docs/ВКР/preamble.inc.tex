\sloppy

% Настройки стиля ГОСТ 7-32
% Для начала определяем, хотим мы или нет, чтобы рисунки и таблицы нумеровались в пределах раздела, или нам нужна сквозная нумерация.
% \EqInChapter % формулы будут нумероваться в пределах раздела
% \TableInChapter % таблицы будут нумероваться в пределах раздела
% \PicInChapter % рисунки будут нумероваться в пределах раздела

% Добавляем гипертекстовое оглавление в PDF
% \usepackage[
% bookmarks=true, colorlinks=true, unicode=true,
% urlcolor=black,linkcolor=black, anchorcolor=black,
% citecolor=black, menucolor=black, filecolor=black,
% ]{hyperref}

% Изменение начертания шрифта --- после чего выглядит таймсоподобно.
% apt-get install scalable-cyrfonts-tex

% \usepackage{cyrtimespatched}

\usepackage{graphicx}   % Пакет для включения рисунков
\graphicspath{ {./images/} }

% С такими оно полями оно работает по-умолчанию:
% \RequirePackage[left=20mm,right=10mm,top=20mm,bottom=20mm,headsep=0pt]{geometry}
% Если вас тошнит от поля в 10мм --- увеличивайте до 20-ти, ну и про переплёт не забывайте:
\geometry{top=20mm}
\geometry{bottom=20mm}
\geometry{right=15mm}
\geometry{left=30mm}


% Пакет Tikz
\usepackage{tikz}
\usetikzlibrary{arrows,positioning,shadows}

% ячейки в несколько строчек
\usepackage{multirow}

% itemize внутри tabular
\usepackage{paralist,array}

\usepackage{hyperref}

% Стили аннотаций
\usepackage{caption}
\captionsetup{font={stretch=1}}

% Размер шрифтов для заголовков
\usepackage{titlesec}
\titleformat{\chapter}[block]{\fontsize{14}{16}\selectfont\bfseries}{\thechapter}{1em}{}
\titleformat{\section}[block]{\fontsize{14}{16}\selectfont\bfseries}{\thesection}{1em}{}
\titleformat{\subsection}[block]{\fontsize{14}{16}\selectfont\bfseries}{\thesubsection}{1em}{}
\titleformat{\subsubsection}[block]{\fontsize{14}{16}\selectfont\bfseries}{\thesubsubsection}{1em}{}

\titlespacing*{\chapter}{0pt}{0pt}{0pt}
% \titlespacing*{\section}
%   {0pt}{3.5ex plus 1ex minus .2ex}{2.3ex plus .2ex}
% \titlespacing*{\subsection}{0pt}{3.5ex plus 1ex minus .2ex}{2.3ex plus .2ex}
% \titlespacing*{\subsubsection}{0pt}{3.5ex plus 1ex minus .2ex}{2.3ex plus .2ex}


% Отсупы в перечислениях
\usepackage{enumitem}
\usepackage[russian]{babel}
\AddEnumerateCounter{\asbuk}{\@asbuk}{а}
\usepackage{enumitem}
\setlist[enumerate]{
    label=\asbuk*),
    leftmargin=0.67cm, 
}

% Нумерации в библио c точками
\usepackage{etoolbox}
\makeatletter
\patchcmd{\@biblabel}{#1}{#1.}{}{}
\makeatother

% Елочка в содержании
\usepackage{tocloft}
\setlength{\cftchapindent}{0pt}      % глава без отступа
\setlength{\cftsecindent}{1em}     % отступ для \section
\setlength{\cftsubsecindent}{1.75em}    % отступ для \subsection
\setlength{\cftsubsubsecindent}{4.5em} % если используете \subsubsection

\setlength{\cftsecnumwidth}{2.5em}      % ширина для номера section
\setlength{\cftsubsecnumwidth}{3.5em}   % ширина для номера subsection
\setlength{\cftsubsubsecnumwidth}{4.5em} % ширина номера subsubsection
