\myChapterSmall{ВВЕДЕНИЕ}

В настоящее время геоданные являются неотъемлемой частью многих высоконагруженных систем, таких как поисковые системы, социальные сети, картографические сервисы, геосоциальные карты и другие. Однако обработка и хранение большого объема геоданных может стать проблемой для разработчиков. Для решения данной проблемы были разработаны методы геопоиска и пространственные индексы, которые позволяют эффективно работать с геоданными в высоконагруженных системах.

Цель данной дипломной работы — провести сравнительный анализ применения алгоритмов геопоиска и пространственных индексов в высоконагруженных системах. В работе будут рассмотрены основные принципы работы указанных методов, их преимущества и недостатки. Также будет проведено сравнение производительности данных методов на различных наборах геоданных и на различных условиях задач.

Под высоконагруженной системой подразумевается система, в которой нет возможности бесконечно масштабировать систему вертикально и разработчикам приходится оптимизировать существующие алгоритмы и подходы. Зачастую такая система имеет большое количество запросов как на чтение, так и на запись, а общее количество запросов к системе в секунду (RPS) превышает несколько тысяч. 

Основной проблемой работы с геоданными в высоконагруженных системах является неочевидность в выборе структур и методов хранения данных. Так, например, классический индекс B-Tree нельзя использовать при обращении к кортежам координат, потому что B-Tree умеет работать только со скалярными данными. При этом, для работы с геоданными существует большое количество индексов, например, R-Tree, KD-Tree, Quadtree, каждый из которых имеет свои плюсы и минусы. Помимо этого имеется проблема сериализации и десериализации данных, которая заключается в том, что результаты анализа поиска, запросы на поиск и сами данные точек необходимо передавать по сети в как можно меньшем объеме и размере (в данном случае под объемом подразумевается количество передаваемых данных, а под размером — вес в байтах).

Результаты данной работы могут быть полезны для разработчиков высоконагруженных систем, работающих с геоданными, а также для специалистов в области геоинформатики.

