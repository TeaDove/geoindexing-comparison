%% Преамбула TeX-файла

% 1. Стиль и язык
\documentclass[utf8x, 14pt, openany, oneside]{G7-32} % Стиль (по умолчанию будет 14pt)
\usepackage[T2A]{fontenc}
\usepackage[russian]{babel}
\usepackage{float}
\usepackage{pdfpages}
\usepackage{tabularx}
\usepackage{centernot}
\usepackage{tabularx}
\usepackage[utf8]{inputenc}
% Остальные стандартные настройки убраны в preamble.inc.tex.
\sloppy

% Настройки стиля ГОСТ 7-32
% Для начала определяем, хотим мы или нет, чтобы рисунки и таблицы нумеровались в пределах раздела, или нам нужна сквозная нумерация.
% \EqInChapter % формулы будут нумероваться в пределах раздела
% \TableInChapter % таблицы будут нумероваться в пределах раздела
% \PicInChapter % рисунки будут нумероваться в пределах раздела

% Добавляем гипертекстовое оглавление в PDF
% \usepackage[
% bookmarks=true, colorlinks=true, unicode=true,
% urlcolor=black,linkcolor=black, anchorcolor=black,
% citecolor=black, menucolor=black, filecolor=black,
% ]{hyperref}

% Изменение начертания шрифта --- после чего выглядит таймсоподобно.
% apt-get install scalable-cyrfonts-tex

% \usepackage{cyrtimespatched}

\usepackage{graphicx}   % Пакет для включения рисунков
\graphicspath{ {./images/} }

% С такими оно полями оно работает по-умолчанию:
% \RequirePackage[left=20mm,right=10mm,top=20mm,bottom=20mm,headsep=0pt]{geometry}
% Если вас тошнит от поля в 10мм --- увеличивайте до 20-ти, ну и про переплёт не забывайте:
\geometry{top=20mm}
\geometry{bottom=20mm}
\geometry{right=15mm}
\geometry{left=30mm}


% Пакет Tikz
\usepackage{tikz}
\usetikzlibrary{arrows,positioning,shadows}

% ячейки в несколько строчек
\usepackage{multirow}

% itemize внутри tabular
\usepackage{paralist,array}

\usepackage{hyperref}

% Стили аннотаций
\usepackage{caption}
\captionsetup{font={stretch=1}}

% Размер шрифтов для заголовков
\usepackage{titlesec}
\titleformat{\chapter}[block]{\fontsize{14}{16}\selectfont\bfseries}{\thechapter}{1em}{}
\titleformat{\section}[block]{\fontsize{14}{16}\selectfont\bfseries}{\thesection}{1em}{}
\titleformat{\subsection}[block]{\fontsize{14}{16}\selectfont\bfseries}{\thesubsection}{1em}{}
\titleformat{\subsubsection}[block]{\fontsize{14}{16}\selectfont\bfseries}{\thesubsubsection}{1em}{}

\titlespacing*{\chapter}{0pt}{0pt}{0pt}
% \titlespacing*{\section}
%   {0pt}{3.5ex plus 1ex minus .2ex}{2.3ex plus .2ex}
% \titlespacing*{\subsection}{0pt}{3.5ex plus 1ex minus .2ex}{2.3ex plus .2ex}
% \titlespacing*{\subsubsection}{0pt}{3.5ex plus 1ex minus .2ex}{2.3ex plus .2ex}


% Отсупы в перечислениях
\usepackage{enumitem}
\usepackage[russian]{babel}
\AddEnumerateCounter{\asbuk}{\@asbuk}{а}
\usepackage{enumitem}
\setlist[enumerate]{
    label=\asbuk*),
    leftmargin=0.67cm, 
}

% Нумерации в библио c точками
\usepackage{etoolbox}
\makeatletter
\patchcmd{\@biblabel}{#1}{#1.}{}{}
\makeatother

% Елочка в содержании
\usepackage{tocloft}
\setlength{\cftchapindent}{0pt}      % глава без отступа
\setlength{\cftsecindent}{1em}     % отступ для \section
\setlength{\cftsubsecindent}{1.75em}    % отступ для \subsection
\setlength{\cftsubsubsecindent}{4.5em} % если используете \subsubsection

\setlength{\cftsecnumwidth}{2.5em}      % ширина для номера section
\setlength{\cftsubsecnumwidth}{3.5em}   % ширина для номера subsection
\setlength{\cftsubsubsecnumwidth}{4.5em} % ширина номера subsubsection


% Настройки листингов.
\include{listings.inc}

% Полезные макросы листингов.
% Любимые команды
\newcommand{\Code}[1]{\textbf{#1}}
\newcommand{\eqspace}[0]{\setlength\abovedisplayskip{3.2ex}\setlength\belowdisplayskip{3.2ex}\setlength\abovedisplayshortskip{3.2ex}\setlength\belowdisplayshortskip{3.2ex}}

\newcommand{\myChapterParts}[2]{
  \newpage
  \noindent
  \vspace{1em}
  \textbf{\centerline{#1}}
  \textbf{\centerline{#2}}
  \addcontentsline{toc}{chapter}{#1 #2}
}

\newcommand{\myChapterStar}[1]{%
  \newpage
  \noindent
  \textbf{\centerline{#1}}\newline
}

\newcommand{\myChapterSmall}[1]{%
  \newpage
  \noindent
  \textbf{\centerline{#1}}\newline
  \addcontentsline{toc}{chapter}{#1}
}

\newcommand{\myChapter}[1]{%
  \newpage
  \par
  \textbf{#1}
  \newline
  \addcontentsline{toc}{chapter}{#1}
}






%%% Local Variables:
%%% mode: latex
%%% TeX-master: "rpz"
%%% Stop:


\begin{document}

\fontsize{14}{16}\selectfont

\frontmatter % выключает нумерацию ВСЕГО; здесь начинаются ненумерованные главы: реферат, введение, глоссарий, сокращения и прочее.

% Команды \breakingbeforechapters и \nonbreakingbeforechapters
% управляют разрывом страницы перед главами.
% По-умолчанию страница разрывается.

% \nobreakingbeforechapters

% \includepdf{pdf/title-page.pdf}

\includepdf[pages={1,2}]{pdf/task.pdf}
\setcounter{page}{4}
\Referat %% Список обозначений и сокращений в тексте
%%Тут должен быть реферат по типу:

ВКР 59 с., 24 рис., 4 ф-л, 1 табл., 20 источн., 2 прил. %%это поменять под себя

\noindent ДРЕВОВИДНЫЕ СТРУКТУРЫ ДАННЫХ, АБСТРАКТНЫЕ СТРУКТУРЫ ДАННЫХ, ИНДЕКСЫ, ПРОСТРАНСТВЕННЫЕ ИНДЕКСЫ, R-TREE, KD-TREE, GEOHASH, H3, B-TREE, ГЕОПОИСК

%%Ключевые слова от 5 до 15 слов или словосочетаний КАПСОМ. ТОЧКУ в конце  НЕ СТАВИТЬ 

Объект исследования: пространственные индексы и алгоритмы поиска по геоданным. 

Цель работы: разработать программное обеспечение для практического анализа популярных алгоритмов геопоиска и пространственных индексов, которые потенциально могут использоваться в высоконагруженных системах, а также математический (теоретический) анализ указанных методов и алгоритмов. Доработка существующих и разработка новых алгоритмов под указанные задачи.

Методы проведения работы: дискретная математика, теория графов, анализ производительности индексов. 

Результат работы: система нагрузочного тестирования индексов. Новые индексы, оптимизация уже существующих индексов. Оценка качества индексов. 

Область применения результатов: бекенд разработка, СУБД.


%%Предметом исследования ВКР является: Цель работы..., В процессе работы проводились ..., В результате работы впервые были ...
%%Внимание!
%%Текст Реферата не более 1 страницы. Должен отражать:
%%- объект исследования;
%%- цель работы;
%%- методы проведения работы;
%%- результаты работы и их новизну;
%%- область применения результатов.+

%%% Local Variables:
%%% mode: latex
%%% TeX-master: "rpz"
%%% End:

\ReferatEN
%%Полный перевод текста РЕФЕРАТА на английский язык.

FQW 62 p., 24 fig., 4 f-l, 1 tabl., 20 sources, 2 app.

\noindent TREE-LIKE DATA STRUCTURES, ABSTRACT DATA TYPES, INDEXES, SPACIAL INDEXES, R-TREE, KD-TREE, GEOHASH, H3, B-TREE, GEOSEARCH

%%Ключевые слова от 5 до 15 слов или словосочетаний КАПСОМ. ТОЧКУ в конце  НЕ СТАВИТЬ 

Object of research: spacial indexes and search algorithms on geo data.

The purpose of the work: develop software for practical analysis of popular geo algorithms and spacial indexes, which could potentially be used in high loaded systems. Mathematical (theoretical) analysis of the reported methods and algorithms. Modification of the existing algorithms, development of new algorithms. 

Methods of work: discrete mathematics, graph theory, performance analysis of indexes.

The result of the work: loadtesting system for indexes. New indexes, optimizations of existing indexes. Performance scoring of indexes.

Scope of the results: backend development, DBMS



%%% Local Variables:
%%% mode: latex
%%% TeX-master: "rpz"
%%% End:


\myChapterStar{СОДЕРЖАНИЕ}
\vspace{-2em}
\tableofcontents

\clearpage
\onecolumn



  \Defines % Необходимые определения. Вряд ли понадобться

В настоящей выпускной квалификационной работе применяют следующие термины с соответствующими определениями.

\vspace{1em}

\noindent
\begin{tabularx}{\textwidth}{m{4cm} m{0.5cm} X}
Индекс & - & абстрактная структура данных, создаваемая с целью повышения операций поиска по заданному массиву за счёт повышения затрат на хранение. Примеры: в СУБД: b-tree, в Python: dict, в C++: map \\
Пространственный индекс & - & индекс, построенный для ускорения поиска по геоданным, например, для решения задачи KNN или поиска в прямоугольнике. Примеры: R-Tree, KD-Tree и т.д. \\
\end{tabularx}

%%% Local Variables:
%%% mode: latex
%%% TeX-master: "rpz"
%%% End:

\Abbreviations %% Список обозначений и сокращений в тексте
В настоящей КНИР применяют следующие сокращения и обозначения:

\noindent СУБД --- Система управления базами данных

\noindent БД --- База данных

\noindent SQL (Structured Query Language) --- Язык структурированных запросов. Декларативный язык запросов для взаимодействия с СУБД. Используется в таких базах данных как: Postgres, MySQL, MSSQL, SQLite и тд.

\noindent RPS (Requests per second) --- Запросов за секунду. Метрика оценки нагруженности системы по количество запросов от клиентов серверу в секундах.

\noindent KNN (k-nearest neighbors) - К-ближайщих соседей, задача поиска К-ближайших соседей к точке А по метрике D.

%%% Local Variables:
%%% mode: latex
%%% TeX-master: "rpz"
%%% Stop:

\myChapterSmall{ВВЕДЕНИЕ}

В настоящее время геоданные являются неотъемлемой частью многих высоконагруженных систем, таких как поисковые системы, социальные сети, картографические сервисы, геосоциальные карты и другие. Однако обработка и хранение большого объема геоданных может стать проблемой для разработчиков. Для решения данной проблемы были разработаны методы геопоиска и пространственные индексы, которые позволяют эффективно работать с геоданными в высоконагруженных системах.

Цель данной дипломной работы — провести сравнительный анализ применения алгоритмов геопоиска и пространственных индексов в высоконагруженных системах. В работе будут рассмотрены основные принципы работы указанных методов, их преимущества и недостатки. Также будет проведено сравнение производительности данных методов на различных наборах геоданных и на различных условиях задач.

Под высоконагруженной системой подразумевается система, в которой нет возможности бесконечно масштабировать систему вертикально и разработчикам приходится оптимизировать существующие алгоритмы и подходы. Зачастую такая система имеет большое количество запросов как на чтение, так и на запись, а общее количество запросов к системе в секунду (RPS) превышает несколько тысяч. 

Основной проблемой работы с геоданными в высоконагруженных системах является неочевидность в выборе структур и методов хранения данных. Так, например, классический индекс B-Tree нельзя использовать при обращении к кортежам координат, потому что B-Tree умеет работать только со скалярными данными. При этом, для работы с геоданными существует большое количество индексов, например, R-Tree, KD-Tree, Quadtree, каждый из которых имеет свои плюсы и минусы. Помимо этого имеется проблема сериализации и десериализации данных, которая заключается в том, что результаты анализа поиска, запросы на поиск и сами данные точек необходимо передавать по сети в как можно меньшем объеме и размере (в данном случае под объемом подразумевается количество передаваемых данных, а под размером — вес в байтах).

Результаты данной работы могут быть полезны для разработчиков высоконагруженных систем, работающих с геоданными, а также для специалистов в области геоинформатики.



\myChapter{1 АНАЛИТИЧЕСКИЙ ОБЗОР ЛИТЕРАТУРЫ}
\label{cha:analysis}

В данной работе будут анализироваться только 3 задачи поиска по геоданным. Пусть дано множество точек, представляющих собой пару широта-долгота, отображающих некий объект на поверхности земли, например, самокат, пользователя, дом. 

Необходимо произвести операции:
\par а) BBox(bottomLeft Point, upperRight Point) Points - поиск всех объектов, которые находятся внутри прямоугольника, определяемого нижним левым углом (bottomLeft) и правым верхним углом (upperRight). Возвращается массив найденных точек;
\par б) KNN(p Point, k int) Points - поиск k ближайших объектов от точки p. Возвращается массив найденных точек;
\par в) Insert(p Point) - вставка нового элемента (точки) в индекс. 

Важно отметить, что зачастую для прикладных задач также требуется решить задачу поиска в круге: найти все точки X, находящиеся на заданном расстоянии R от точки Y. Эта задача не рассматривается в данной работе из-за того, что на части индексов она не может быть решена оптимально. Например, в дереве R-tree. При этом её все равно можно решить через задачу BBox: 
\par а) решить задачу BBox для квадрата, описывающего заданный круг;
\par б) простым перебором отсечь точки, которые не входят в заданный круг.

Задача BBox может быть также сведена к задаче поиска в круге:
\par а) решить задачу поиска в круге для круга, описывающего прямоугольник;
\par б) простым перебором отсечь точки, которые не входят в заданный прямоугольник.

\section{1.1 Расстояние между двумя точками}
Указанные выше задачи требуют определить понятие расстояния между двумя точками на сфере земли. Для этого введем понятие геоточки --- это точка, представляющая собой кортеж A(x, y), где x и y - широта и долгота соответственно. 
Расстояние между двумя точками есть наименьшая прямая на плоскости земного шара, при этом ее вычисление может быть разным в зависимости от рассматриваемой задачи. \\
Определим задачу. Даны геоточки $A(x_1, y_1), B(x_2, y_2)$. Требуется найти расстояние $d(A, B)$.

\subsection{1.1.1 Евклидово расстояние}
В некоторых системах, например, в расширении Postgis для СУБД Postgres для типа Geometry, используется евклидово расстояние.
  \newline
\begin{equation} \label{eq}
\
\ d(A, B)=sqrt((x_1-x_2)^2+(y_1-y_2)^2).
\
\end{equation}
  \newline
Такой подход работает корректно только в части случаев, так, например, он выдаст правильный результат с точностью до 5 метров в случае поиска расстояния в пределах Москвы, но расстояние в Африке будет отличаться в 2-3 раза относительно более строгих подходов.  
При этом важно отметить, что данная дистанция отвечает аксиомам метрики, поэтому ее корректно использовать при сравнении расстояний между двумя точками, например, при решении задачи KNN (К-ближайших соседей). 

\subsection{1.1.2 Расстояние гаверсинуса}

Расстояние гаверсинуса — это способ определения расстояния между двумя точками на поверхности Земли, учитывающий её кривизну. Он используется в геопоиске для определения расстояния между заданной точкой и объектами в заданном радиусе.

Формула расстояния гаверсинуса выглядит следующим образом.
    \newline
\begin{equation}\label{eq}
    \begin{aligned}
        d(A, B) = \\
        = 2R \arcsin\left(sqrt(\frac{1 - \cos\left(\Delta x \right) + \cos x_1 \cdot \cos x_2 \cdot \left(1 - \cos\left(\Delta y\right)\right)}{2})\right),
    \end{aligned}
\end{equation}
\par\vspace{1em}
\par\vspace{1em}
\noindent
где d - расстояние между двумя точками в километрах, R - радиус Земли (приблизительно 6371 км), $x_1$ и $x_2$ - широты двух точек в радианах, $y_1$ и $y_2$ - долготы двух точек в радианах.

Эта формула позволяет определить расстояние между точками с точностью до нескольких метров, что делает ее очень полезной для геопоиска. Однако она может быть достаточно ресурсоёмкой при работе с большими объемами геоданных, что делает ее медленнее по сравнению с Евклидовым расстоянием. 

\subsection{1.1.3 Геодезическое расстояние}
Геодезическое расстояние — это расстояние между двумя точками на поверхности Земли, измеренное вдоль кратчайшей линии (геодезической линии) между этими точками. Геодезическая линия — это кривая на поверхности Земли, которая имеет наименьшую длину между двумя точками.

Геодезическое расстояние учитывает кривизну Земли и может отличаться от расстояния гаверсинуса, особенно на больших расстояниях и при использовании разных моделей формы Земли.

Формула вычисления геодезии.
\begin{equation} \label{eq}
\ d(A, B) = R \cdot \arccos(\sin(x_1) \cdot \sin(x_1) + \cos(x_1) \cdot \cos(x_2) \cdot \cos(y_2 - y_1).
\end{equation}
\par
Таким образом, при выборе формулы поиска расстояния надо учитывать требования. Например, при разработке приложения кикшеринга, то есть аренды электронных самокатов, формулы можно комбинировать. На запросах, которые возвращают клиентам ближайшие самокаты от их текущей геопозиции по радиусу можно использовать метрику Евклида, так как ошибку даже на 20-30 метров можно нивелировать поиском по большему радиусу.  
При этом, в ситуации, когда требуется построить аналитику данных, лучше использовать расстояние геодезии, потому что в данных задачах крайне важна точность результата, а разница в скорости решения не так принципиальна. 

\section{1.2 Пространственные индексы}
Как и с задачами поиска по массиву скаляров, для поиска по геоточкам используются специальные структуры данных, которые позволяют оптимизировать указанные операции. 

Данные структуры можно разбить на 2 типа: древовидные и хэши со скалярным индексом. К первому виду относятся: R-tree, VP-tree, BSP-tree, Quadtree, KD-tree. К хэшам относятся: Geohash, S2 Geometry и Uber H3.

Рассмотрим принцип работы древовидных структур на примере KD-tree, самого простого для понимания человеком пространственного индекса.

\subsection{1.2.1 Дерево KD-tree}
KD-tree представляет собой дерево, позволяющее производить операции поиска в N-мерном пространстве. Рассмотрим двухмерное KD-tree, которое также можно назвать K-2 tree.
Сама структура KD-tree является обычным бинарным деревом, как показано на рисунке 1.
\par\vspace{1em}
\begin{figure}[H]
    \centering
    \includegraphics[scale=0.2]{kd-tree.jpg}
    \caption{Слева — пример визуализации разбиения плоскости через KD-tree. Справа --- результат построения дерева}
\end{figure}

Алгоритм построения KD-tree довольно прост\cite{nnq}:
\par а) происходит поиск центральной точки, то есть той точки, которая будет находиться на суммарно меньшем расстоянии от всех точек. Данная точка ставится в корень дерева KD-tree;
\par б) далее плоскость разбивается на 2 части по вертикальной оси;
\par в) слева ищется средняя точка, то есть та точка, которая по оси абсцисс (широте) находится на суммарно меньшем расстоянии до остальных. Данная точка записывается в левого ребёнка корня дерева;
\par г) аналогичная процедура повторяется справа;
\par д) аналогичная процедура повторяется для вновь созданных полотен, но уже с осью ординат (долготой);
\par е) данные процедуры повторяются со всеми точками.

\subsection{1.2.2 Поиск KNN в KD-tree}

Процесс поиска K ближайших соседей по KD-tree заключается в следующих шагах:
\par а) строим KD-tree из набора точек;
\par б) находим ближайшую к заданной точке точку в дереве. Для этого спускаемся по дереву, сравнивая координаты заданной точки и текущей точки в узле дерева. Если координата текущей точки больше или равна координате заданной точки, то спускаемся в левое поддерево, иначе - в правое. При этом сохраняем расстояние между текущей точкой и заданной точкой;
\par в) добавляем найденную точку в список ближайших соседей;
\par г) проверяем, есть ли еще точки в дереве, которые могут быть ближе к заданной точке, чем уже найденные соседи. Для этого проверяем расстояние между заданной точкой и границей текущего поддерева (это можно сделать, используя формулу расстояния между точками). Если это расстояние меньше, чем расстояние до самого дальнего найденного соседа, то нужно проверить и другое поддерево;
\par д) повторяем шаги 2-4 для всех точек в дереве, пока не найдем K ближайших соседей или не пройдем по всему дереву.

В результате получаем список K ближайших соседей заданной точки. 

\subsection{1.2.3 Поиск в круге в KD-tree}
Процесс поиска K ближайших точек и всех точек в заданном радиусе очень похож на процесс поиска по бинарному дереву за тем исключением, что сравнение по четным нодам идет по широте, а по нечетным — по долготе. 

Процесс заключается в следующих шагах:
\par а) строим KD-дерево на основе набора геоданных;
\par б) ищем листовой узел дерева, который содержит заданную точку. Для этого начинаем с корневого узла и спускаемся по дереву, выбирая каждый раз ту часть пространства, которая содержит заданную точку;
\par в) находим все точки, которые находятся в заданном круге с центром в заданной точке и радиусом R. Для этого проверяем каждую точку в листовом узле и всех его родительских узлах на расстояние до заданной точки. Если расстояние меньше или равно R, то добавляем точку в список найденных точек;
\par г) возвращаем список точек, которые находятся в заданном круге.


\subsection{1.2.4 Дерево Quadtree}
Quadtree - это пространственный индекс, который представляет пространство в виде дерева. Крайне похож на KD-tree за тем исключением, что пространство делится на равные подпространства. Представление дерева на плоскости и в памяти ЭВМ показано на рисунке 2. 
\par\vspace{1em}
\begin{figure}[H]
    \centering
    \includegraphics[scale=0.2]{quadtree.png}
    \caption{Слева --- пример визуализации разбиения плоскости через Quadtree. Справа — результат построения дерева}
\end{figure}
  
Процесс поиска точек в заданном круге по Quadtree:
\par а) конвертируем заданные координаты в координаты узла Quadtree. Это делается путем разбиения пространства на сетку и назначения каждой ячейке уникальных координат узла Quadtree;
\par б) определяем узлы Quadtree, которые находятся внутри заданного круга. Для этого используем формулу гаверсинуса для вычисления расстояния между заданными координатами и каждым узлом Quadtree. Если расстояние меньше или равно радиусу круга, то добавляем этот узел в список;
\par в) ищем все точки, которые соответствуют найденным узлам Quadtree. Для этого используем индекс, который связывает каждый узел Quadtree с набором геоданных. Ищем все точки в наборе, которые соответствуют найденным узлам Quadtree;
\par г) возвращаем список точек, которые находятся в заданном круге.


\subsection{1.2.5 Дерево R-tree}
R-tree - это пространственный индекс, который используется для хранения и поиска объектов в пространстве\cite{guttmanRtree}. Как показано на рисунке 3, R-tree представляет собой дерево, где каждый узел представляет собой прямоугольник, содержащий объекты. Этот индекс позволяет быстро находить объекты, которые находятся в заданном прямоугольнике или близко к нему. Логика работы данного индекса крайне похожа на логику индекса B-Tree, но для координат.
\par\vspace{1em}
\begin{figure}[H]
    \centering
    \includegraphics[scale=0.2]{R-tree.png}
    \caption{Сверху — пример визуализации разбиения плоскости через R-tree. Снизу - результат построения дерева}
\end{figure}
  
Процесс поиска объектов в заданном прямоугольнике по R-tree:
\par а) конвертируем заданный прямоугольник в формат R-tree. Это делается путем определения минимального и максимального значения координат для каждого измерения;
\par б) ищем узлы R-tree, которые пересекаются с заданным прямоугольником. Для этого используем алгоритм пересечения прямоугольников, который позволяет быстро определить, какие узлы R-tree находятся внутри или пересекаются с заданным прямоугольником;
\par в) ищем все объекты, которые соответствуют найденным узлам R-tree. Для этого используем индекс, который связывает каждый узел R-tree с набором объектов. Ищем все объекты в наборе, которые соответствуют найденным узлам R-tree;
\par г) возвращаем список объектов, которые находятся в заданном прямоугольнике.


Важно отметить, что данная структура используется в большом количестве популярных СУБД: 
\par а) PostGIS - Расширение для СУБД PostgreSQL;
\par б) Redis - Key-value база данных;
\par в) Tile38 - Аналог Redis с углублением в работу с геоданными.

Таким образом данный индекс выигрывает по популярности у разработчиков. 

\subsection{1.2.6 Дерево R*-tree}
R*-tree представляет собой оптимизацию балансировки R-tree\cite{beckmannRStarTree}, благодаря которой операции поиска становятся быстрее, но при этом операция вставки замедляется за счет необходимости перестройки дерева. Необходимо подчеркнуть, что в указанных выше СУБД используется именно оригинальная версия дерева --- R-tree\cite{sunRStartree}. Данный факт обязует на практике протестировать разницу между деревьями и проанализировать причины такого выбора\cite{fedorovskieRTree}. 

\subsection{1.2.7 Индекс Bruteforce}
Для наиболее честного анализа качества работы индексов, требуется использовать контрольный индекс, который можно именовать как перебор (Bruteforce). Данный индекс представляет собой простой динамический массив, в который сохраняются все заданные точки. Данный массив можно использовать как самое «плохое» решение «в лоб», иными словами, если какой-то алгоритм или индекс работает хуже, чем перебор - значит данный индекс либо разработан неправильно, либо имеет слишком критические недостатки. 

Процесс поиска ближайших соседей:
\par а) вычислить матрицу расстояний от заданной точки до каждой точки массива;
\par б) отсортировать массив по расстояниям используя указанную матрицу;
\par в) взять первые K элементов.

Процесс поиска в прямоугольнике:
\par а) последовательно итерироваться по всем элементам массива и добавлять в результирующий массив все точки, чьи широты и долготы входят в прямоугольник.

\subsection{1.2.8 Дерево BSP-tree}
BSP-tree (Binary Space Partitioning tree) - это пространственный индекс, который используется для разбиения пространства на бинарные подпространства\cite{liuGBTree}. BSP-tree представляет собой бинарное дерево, где каждый узел представляет собой гиперплоскость, которая разделяет пространство на две части: левую и правую.

Процесс поиска ближайших соседей по BSP-tree заключается в следующих шагах:
\par а) строим BSP-tree из набора точек. Для этого выбираем случайную гиперплоскость и разбиваем набор точек на две группы: те, что находятся по одну сторону от гиперплоскости, и те, что находятся по другую сторону;
\par б) ищем ближайшую точку к заданной точке. Для этого начинаем с корня BSP-tree и рекурсивно спускаемся по дереву, выбирая поддерево, которое содержит более близкие к заданной точке точки. Если расстояние от гиперплоскости до заданной точки меньше, чем текущее расстояние до ближайшей точки, то ищем ближайшую точку в этом поддереве;
\par в) после этого проверяем, есть ли еще поддеревья, которые могут содержать более близкие точки, и продолжаем поиск.

\subsection{1.2.9 Дерево VP-tree}
VP-tree (Vantage Point tree) - это пространственный индекс, который используется для быстрого поиска ближайших соседей в многомерном пространстве. VP-tree представляет собой бинарное дерево, где каждый узел представляет собой точку-центр (опорная точка, как показано на рисунке 4). От указанной точки строятся два поддерева: левое поддерево содержит все точки, которые находятся внутри заданного радиуса от центра, а правое поддерево содержит все точки, которые находятся за пределами этого радиуса. Является разновидностью дерева BSP.

\par\vspace{1em}
\begin{figure}[H]
    \centering
    \includegraphics[scale=0.8]{vp-tree.png}
    \caption{Разбиение пространства на 3 подпространства через VP-Tree}
\end{figure}
  
Процесс поиска ближайших соседей по VP-tree заключается в следующих шагах:
\par а) строим VP-tree из набора точек. Для этого выбираем случайную опорную точку и разбиваем набор точек на две группы: те, что находятся ближе к центру, и те, что находятся дальше от него;
\par б) ищем ближайшую точку к заданной точке. Для этого начинаем с корня VP-tree и рекурсивно спускаемся по дереву, выбирая поддерево, которое содержит более близкие к заданной точке точки. Если расстояние от центра поддерева до заданной точки меньше, чем текущее расстояние до ближайшей точки, то ищем ближайшую точку в этом поддереве;
\par в) после этого проверяем, есть ли еще поддеревья, которые могут содержать более близкие точки, и продолжаем поиск.


\subsection{1.2.10 Иные индексы}
В научном сообществе каждый год публикуются статьи с новыми индексами, например, MPTrie \cite{gantiMPTrie}, STHash\cite{guanSTHash} и другие. Большинство из них решают узконаправленные задачи, либо улучшают уже созданные индексы для работы в определенных условиях. В данной работе указанные индексы не рассматриваются из-за того, что их некорректно было бы сравнивать с R-Tree, KD-Tree и тд. из-за того, что, как уже было сказано, они реализованы под конкретную проблему. 
\section{1.3 Иные структуры данных}

\subsection{1.3.1 Система Geohash}
Geohash (далее также геохеш) - представляет собой бинарной представление координат (широта-долгота)\cite{jiajunGeohash}. Сам геохеш не реализует алгоритмов поиска, поэтому для поиска по геохеше дополнительно используются такие структуры как B-tree, trie, Radix-trie и тд\cite{sahrIndexingSystems}.
Самим геохешом называется строка, закодированная 32 разрядным алфавитом, перевод из 10-ой системы в указанный алфавит указан в таблице 1.
\par\vspace{1em}

\noindent
Таблица 1 --- Алфавит геохеша по основанию 10 и 32
\begin{tabularx}{\textwidth}{ |X|c c c c c c c c| }
 \hline
 Основание 10 & 0 & 1 & 2 & 3 & 4 & 5 & 6 & 7 \\
 Основание 32 & 0 & 1 & 2 & 3 & 4 & 5 & 6 & 7 \\
  \hline
 Основание 10 & 8 & 9 & 10 & 11 & 12 & 13 & 14 & 15 \\
 Основание 32 & 8 & 9 & b & c & d & e & f & g \\
  \hline
 Основание 10 & 16 & 17 & 18 & 19 & 20 & 21 & 22 & 23  \\
 Основание 32 & h & j & k & m & n & p & q & r \\
  \hline
 Основание 10 & 24 & 25 & 26 & 27 & 28 & 29 & 30 & 31 \\
 Основание 32 & s & t & u & v & w & x & y & z \\
  \hline
\end{tabularx}
\par\vspace{1em}
\par\vspace{1em}

Данная строка однозначно декодируется в кортеж геокоординат с точностью, зависящей от количества символов в строке. Примеры:
\par а) строка \texttt{ucft} дегодируется в прямоугольник площадью примерно 800 $ km^2 $ и центром в (55.81, 37.44);
\par б) строка \texttt{ucft943} имеет тот же центр - (55.8236, 37.3116), но меньшую площадь $23409 m^2$
Как можно наблюдать, чем выше количество символов, используемых в геохеше, чем выше точность получаемых координат, но при этом выше затрачиваемая память\cite{sidorovGeohash}.
В данной работе не будет детально описываться процесс формирования геохеша за исключением базового принципа: 

Сфера земли разбивается на практически равные прямоугольники, после чего каждому прямоугольнику присваивается номер в 32х-ричной системе координат, номера присваиваются в порядке змейкой, сначала самый левый-верхний, далее ниже от него, далее справа от самого левого-верхнего и тд, пример разбиения виден на рисунке 5.
\par\vspace{1em}
\begin{figure}[H]
    \centering
    \includegraphics[scale=0.2]{geohash.png}
    \caption{Пример разбиения сферы земли через Geohash на первые 2 префикса}
\end{figure}

Сам по себе Geohash, как и ниже описанные структуры Uber H3 и S2 - не являются индексами\cite{balkicGeohash}, а лишь методами сериализации геоточек. Реализации алгоритмов поиска на этих методах производится в связке с такими структурами как B-tree, Trie и тд. 

\subsection{1.3.2 Система Uber H3}
Uber H3 - это сетка гексагональных ячеек (см. рисунок 6), которая используется сериализации и хранения геоданных в приложениях компании Uber, которая, собственно, и разработала данную систему. Каждая ячейка имеет уникальный идентификатор и может быть использована для определения местоположения объектов. Логика работы Uber H3 аналогична логики Geohash за основным исключением, что в Uber H3 используются шестиугольники, а в Geohash - прямоугольники. 
  \\
\begin{figure}[h]
    \centering
    \includegraphics[scale=0.2]{h3.png}
    \caption{Разбиение сферы земли на шестиугольники при использовании H3}
\end{figure}

Принципиальных отличий между H3 и Geohash\cite{bohuiGeohashH2S2}, помимо, конечно, разницы в форме - нет. И тот и тот формат в конечном итоге кодирует данные в бинарном виде, имеется возможность перевести данные в строковое представление.  

Важно отметить, что между алгоритмами имееются чисто практические отличия:
\par а) geohash более прост в реализации;
\par б) geohash более просто в понимание его человеком;
\par в) geohash поддерживается такими СУБД, как: Postgis, Redis, MongoDB\cite{membreyMongodb};
\par г) H3 имеет очень обширную библиотеку с дополнительными методами;
\par д) H3 поддерживается такими СУБД, как: Clickhouse

\subsection{1.3.3 Система S2 geometry}

S2 Geometry - это библиотека для работы с геометрическими объектами на сфере, разработанная компанией Google. 
\par\vspace{1em}
\begin{figure}[H]
    \centering
    \includegraphics[scale=0.8]{s2-geometry.jpg}
    \caption{Пример разбиения пространства на области с использованием S2 Geometry}
\end{figure}
\par\vspace{1em}

Основой S2 Geometry является иерархическая структура данных, называемая S2 Cell. Как показано на рисунке 7, каждая ячейка S2 Cell представляет собой квадрат на сфере, который может быть разбит на более мелкие квадраты более высокого уровня. Уровень ячейки определяется числом n, которое указывает на количество разбиений квадрата на подквадраты. Чем больше значение n, тем меньше размер каждой ячейки.

S2 не имеет значимых качественных или количественных преимуществ перед Geohash и H3. Также, данный алгоритм имеет довольно скудное представление в СУБД, а также в подключаемых библиотеках в различных языках программирования. 
Из-за перечисленных выше причин данный алгоритм не будет далее рассматриваться в этой работе. 
\section{1.4 Сравнительный анализ}

Чтобы дать корректный ответ на вопрос о том, какую структуру данных стоит использовать, и какие алгоритмы к ней применять, стоит добавить следующие ограничения:
\par а) не будет учитываться время записи на диск. Предполагается, что все операции производятся при использовании оперативной памяти. Данное ограничение обусловлено тем, что операции сериализации и десериализации индексов на постоянное запоминающие устройство в общем случае идентичны для разных индексов;
\par б) требуется учитывать как теоретические показатели, то есть O-нотацию, так и практические результаты.


\subsection{1.4.1 Качественный сравнительный анализ}
Методы геопоиска и пространственных индексов могут использоваться в различных сценариях, включая навигационные приложения, сервисы доставки еды и товаров, системы мониторинга транспорта и т.д.\cite{heHBase}

Как в задачах анализа данных, так и при разработке операционных сервисов, лидирует индекс R-Tree. Скорее всего это связано со следующими причинами:
\par а) используется в популярном аддоне PostGIS для Postgres;
\par б) эффективен;
\par б) имеет множество методов помимо BBox, KNN и тд.

Таким образом, в общем случае перед разработчиком и не встает вопрос о выборе индекса. Но, важно отметить, что есть множество ситуаций, из-за которых выбор может пасть на другие методы и индексы, например, требования по кластеризации. 

Выбор методов геопоиска и пространственных индексов зависит от конкретной задачи и сценария использования. Необходимо учитывать требования к производительности, точности и эффективности поиска объектов на карте. 

%%% Local Variables:
%%% mode: latex
%%% TeX-master: "rpz"
%%% End:


\chapter{СОДЕРЖАТЕЛЬНАЯ ПОСТАНОВКА ЗАДАЧИ}
\label{cha:statement}

Целью данной работы является разработка программного обеспечения для анализа пространственных индексов в высоконагруженных системах. Данное ПО должно уметь анализировать как минимум предложенные пространственные индексы: R-tree, R*-tree, KD-tree, Geohash+trie, H3+trie.\\
Пусть G - это множество всех точек в индексе, и каждая точка обозначается символом $G_i$, где $G_i \in \mathbb{N}$.
Анализ должен проводится в разрезе трех операций:
\begin{enumerate}
    \item  Insert(p point) - вставка нового точки p в индекс
    \item  KNN(p point, k int) -  поиск множества точек S размера k, находящихся на наименьшем расстояние от точки p. $S=\{G_i | G_i \in G, dist(p, G_i) \leq dist(p, G_j),  \forall G_j \in G, G_j \neq G_i \}, |S| = k$
    \item  RangeSearch(p point, r float) - поиск множества точек S, которых входят в круг, определяющийся центром p и радиусом r. $S=\{G_i | G_i \in G, dist(p, G_i) \leq r)\}$
\end{enumerate}
Где $dist$ - функция расстояния между двумя точками. В данном работе применяется расстояние гаверсинуса. 

Важно отметить, что операция удаления точки не рассматривается из-за того, что в основном она не сильно отличается от операции вставки, а также из-за того, что в высоко нагруженных системах, в основном, применяется подход soft-detele(от англ. мягкое удаление), в котором данные не удаляются, а лишь помечаются в СУБД как неактивные. Также, при проведении анализа данных, в основном, работа происходит с read-only данными, то есть теми данными, которые не меняются. 

Для решения поставленных целей необходимо решить следующие задачи:
\begin{enumerate}
    \item Изучить существующие пространственные индексы
    \item Разработать программное обеспечение для тестирования пространственных индексов.
    \item Протестировать и отладить разработанное программное обеспечение.
    \item Проанализировать результаты работы программного обеспечения. Найти наиболее подходящие структуры под анализируемые задачи.
\end{enumerate}


\section{2.2 Математическая постановка}

В работе будут проверять 3 основные операции над пространственными индексами, которые были подробно описаны в главе «Содержательная постановка задачи»: Insert, KNN, BBox.

Дано:
\par а) $n \in \mathbb{N}$, $n$ - количество элементов в индексе;
\par б) $f(n) = x, x \in  \mathbb{Q}, x > 0$, где $f(n)$ - время выполнения операции в микросекундах.

Для каждой выше указанной операции и для каждой тестируемой структуры требуется вычислить аксиоматическую сложность алгоритма.
\vspace{1em}
\begin{align}
    f_{ax}(n) = O(f(n)).
\end{align}
\par\vspace{1em}

Реализуемое программное обеспечение должно уметь находить $f(n)$ для заданных n и строить график зависимости $f(n)$ от n.
Разработанные и доработанные индексы должны корректно решать поставленные задачи. 

\section{2.3 Разработка}
\label{cha:development}

\subsection{2.3.1 Система тестирования}
Конечный продукт должен представлять собой сервис, который позволяет протестировать реализацию пространственных структур с соответствующими операциями под нагрузкой, а также сами алгоритмы, как взятые готовые решения, так и решения, разработанные в ходе данной работы. 
\begin{enumerate}
    \item Для реализации структур и алгоритмов был выбран язык программирования Golang.  
    \item Для клиентского взаимодействия с сервисом используется HTML и JavaScript.
    \item Для визуализации результатов тестирований используется библиотека ChartsJS.
    \item Для визуализации работы алгоритмов используется библиотека Mapbox.
\end{enumerate}
  \\
\begin{figure}[h]
    \centering
    \includegraphics[scale=0.3]{arch.png}
    \caption{Архитектура системы тестирования}
\end{figure}
  \\
На рисунке 8 визуализирована архитектура решения. Компонент менеджер отвечает за управление и запуск тестов. Компонент \textit{Исполнитель} находится на отдельном от компонента \textit{менеджер} сервере и отвечает за запуск конкретных задач. Сделано это для того, чтобы как можно сильнее изолировать среду запуска и среду анализа и не влиять на результаты запуска тестов какими-либо другими задачами. 


\subsection{2.3.2 Порядок работы с программным обеспечением}
  \\
\begin{figure}[h]
    \centering
    \includegraphics[scale=0.5]{seq.png}
    \caption{Функциональная схема}
\end{figure}
  \\
Процесс использования системы показан на рисунке 9: При открытии страницы с тестами (рисунок 10), пользователь может ознакомиться с доступными индексами и задачами, а также задать количество точек для тестирования. После нажатия кнопки «Старт» начнется тестирование, которое сразу создаст запись с тестом и результатами в виде графиков. 
Пользователь по своему усмотрению может остановить тестирование. Например, если оно выполняется слишком долго. Также он может открыть график с результатами в полном экране, а также скачать его. 
Также имеется страница «Визуализация данных» (рисунок 11), она может быть полезна для разработчиков во время разработки самих индексов, а именно для отладки индексов, визуализации их работы и сравнения с другими индексами. 

\begin{figure}[h]
    \centering
    \includegraphics[scale=0.19]{gui.png}
    \caption{Главный экран}
\end{figure}

\begin{figure}[h]
    \centering
    \includegraphics[scale=0.17]{gui_visualizer.png}
    \caption{Экран визуализации}
\end{figure}

С подробными результатами тестов можно ознакомится в приложение Б.
\section{2.4 Разработанные индексы и алгоритмы}

\subsection{2.4.1 Geohash B-tree}
Во время разработки системы тестирования (рисунок 12), было замечено, что алгоритм \textit{перебора} достаточно хорошо работает на задаче «КНН 25\%». Данная задача требует от алгоритма вернуть K ближайших точек, где $K = \frac{N}{4}$, где N - общее кол-во точек массива. 
  \\«»
\begin{figure}[h]
    \centering
    \includegraphics[scale=0.3]{result_knn_bruteforce.png}
    \caption{Результаты тестирования}
\end{figure}
  \\
Иными словами, алгоритм простого перебора показывает хорошие результаты, когда K близко или сравнимо с N ($K \centernot \ll N$), что было взято за основу алгоритма «Geohash B-tree». 

Устройство разработанного «Geohash B-tree» просто, индекс представляет собой B-Tree\cite{comerBTree}, ключами которого являются геохеши в типе данных uint64, а значениями - массив точек, которые находятся в указанном хеше. Размерность является параметром индекса и фиксируется. Для реализации алгоритмов поиска применяется возможность геохеша кластеризовать пространство земли и находить кластеры, в которых уже далее \textit{перебором} производятся операции поиска по KNN или BBox\cite{gulakovStructured}.
\textit

Итерации поиска в прямоугольнике (рисунок 13)
\begin{enumerate}
    \item Получить геохеши углов, в примере на рисунке 13 - \textit{5} и \textit{y} (если опускать общий префикс)
    \item Получить все точки, что находятся на гранях прямоугольника (\textit{5}, \textit{h}, \textit{j}, \textit{n} и тд.)
    \item Данные точки требуется также проверить на вхождение в прямоугольник через простой перебор
    \item Все точки, что входят во внутрь (\textit{k}, \textit{s}, \textit{m}, \textit{t}) - проверять не надо, они гарантированно входят в прямоугольник
\end{enumerate}
  \\
\begin{figure}[h]
    \centering
    \includegraphics[scale=0.3]{geohash_btree_bbox.png}
    \caption{Алгоритм поиска в прямоугольнике}
\end{figure}
  \\
Итерации поиска ближайшего соседа (рисунок 14)
\begin{enumerate}
    \item Найти геохеш точки P. В примере на рисунке 14 - \textit{ucfv7m}
    \item Получить все точки, в указанном геохеше
    \item По часовой стрелке итерироваться по соседям оригинального геохеша и выгружать точки. (ucfv7m -> ucfv7q -> ucfv7w и тд)
    \item Завершить итерацию, когда кол-во найденных точек будет больше или равно K
    \item Найти наиболее дальшестоящую точку из найденного массива относительно заданной точки P. Пусть расстояние между указанными точками будет R 
    \item Найти все точки в прямоугольнике, с центром в P и сторонами 2R.
    \item Точки в полученном массиве отсортировать по расстоянию и выдать первые K точек
\end{enumerate}
  \\
\begin{figure}[h]
    \centering
    \includegraphics[scale=0.3]{geohash_btree_knn.png}
    \caption{Итеративный проход по соседям}
\end{figure}
  \\
У разработанного алгоритма крайне много преимуществ
\begin{enumerate}
    \item Клиентский. Может быть полностью реализован на стороне клиента, а не СУБД. Если СУБД не поддерживает какие-либо геоиндексы, как например AWS DynamoDB, данный индекс можно использовать для закрытия потребности в геоиндексах. 
    \item Кластерезируемый. В отличие от R-Tree, данный индекс без проблем реализует кластеризацию. Также важно отметить, что B-Tree в данном индексе использует только операции Get и Set, без сравнения и итераций, за счет чего данный индекс можно применять, например, в СУБД Redis, которая нативно поддерживает кластеризацию ключей 
    \item Комбинируемый. В случае необходимости, его можно комбинировать с R-Tree, например, для СУДБ Redis можно хранить в ключах, а в значениях - R-tree через операцию \textit{GEOADD}\cite{redisGeo}
\end{enumerate}
Но также имеется много минусов
\begin{enumerate}
    \item Параметризируемость. На создание индекса требуется выбрать размерность Geohash, она не может быть изменена. Если выбранная размерность будет слишком большой или слишком малой, производительность индекса будет слишком низкой. 
    \item Крайние случаи. Если клиент отправит запрос на задачу KNN, при этом в указанной точке и рядом не будет точек, индекс будет медленно итерироваться по всем ближайшим геохешам.
\end{enumerate}

\myChapter{3 ПРОВЕДЕННЫЕ ИССЛЕДОВАНИЯ. РЕЗУЛЬТАТЫ}
\section{3.1 Результаты тестирования}

Практически на всех задачах такие индексы как \textit{перебор} и \textit{дерево квадрантов} показали плохие результаты (рисунок 15), в свою очередь \textit{R-Tree} и \textit{KD-Tree} - основные финалисты. 
  \\
\begin{figure}[h]
    \centering
    \includegraphics[scale=0.17]{results_knn_1.png}
    \caption{Задача KNN на 1\% точек}
\end{figure}
  \\

При этом разработанный индекс Geohash-B-Tree с правильной параметризацией всегда показывает крайне хорошие результаты (Рисунки 16 и 17)
  \\
\begin{figure}[h]
    \centering
    \includegraphics[scale=0.17]{results_bbox_100.png}
    \caption{Задача BBox на 100 случайных точках}
\end{figure}
  \\
  \\
\begin{figure}[h]
    \centering
    \includegraphics[scale=0.17]{results_knn_100.png}
    \caption{Задача KNN на 100 случайных точках}
\end{figure}
  \\

Таким образом, можно сделать следующие выводы
\begin{enumerate}
    \item Популярные индексы \textit{R-Tree}, \textit{KD-Tree} и другие - в общем случае показывают крайне хорошие результаты 
    \item Перебор часто работает быстрее, чем сложные индексы
    \item Разработанные индексы также показывают крайне хорошие результаты
\end{enumerate}

\section{3.2 Выводы}
По результатам проведенных работ, можно сделать следующие рекомендации относительно выбора индексов
\begin{enumerate}
    \item Если используемая СУБД поддерживает геоиндексы - используйте их, например, из Redis, MongoDB и PostGIS.
    \item Если есть возможность выбрать индекс - рекомендуется выбрать индекс R-Tree, он имеет крайне большое количество методов, реализаций и в общем случае работает крайне хорошо
    \item Если используемая СУБД не поддерживает геоиндексы - их можно реализовать через Geohash + B-Tree. Более сложные реализации (H3 + B-Tree, Geohash + R-Tree) тоже возможны, но не необходимы.
    \item Если требуется кластеризация, то для ключа кластерации можно использовать Geohash + B-Tree. Другие индексы не поддерживают кластеризацию.
\end{enumerate}

\section{3.3 Выводы по практической части}
Разработанное в ходе практической части программное обеспечение позволяет протестировать качество работы тех или иных индексов на разных задачах. Важно отметить, что полученное ПО является легко расширяемым, что позволяет разработчикам его использовать во время разработки, отладки и улучшения указанных индексов.
Преимуществами данной работы являются
\begin{enumerate}
    \item Изолированность. Сервер исполнителя задач физически отделен от сервера менеджера
    \item Наглядность. Пользователь в реальном времени видит результаты, может самостоятельно запрашивать задачи. 
\end{enumerate}
\Conclusion

Таким образом, в рамках выпускной квалификационной работы выполнены следующие пункты:
\begin{enumerate}
    \item Проведен аналитический обзор литературы, в рамках которого были расcмотрены такие индексы и методы как: R-Tree, R*-Tree, KD-Tree, Geohash, H3 и другие;
    \item Cформулированы содержательная и математическая постановки задачи;
    \item В качестве основных инструментариев были выбраны языки программирования Golang и JavaScript;
    \item Проработан клиентский путь, разработана функциональная схема программного обеспечения;
    \item Разработаны дополнительные индексы;
    \item Продемонстрирована система тестирования, а также результаты её работы. Сделаны выводы по указанным результатам;
    \item Основные результаты работы были опубликованы в сборнике тезисов конференции «79-е Дни науки студентов НИТУ «МИСиС» 
    \item Работа представлена на научной конференции кафедры Инженерной кибернетики «79-е Дни науки НИТУ «МИСиС».
\end{enumerate}

% % Список литературы при помощи BibTeX
% Юзать так:
%
% pdflatex rpz
% bibtex rpz
% pdflatex rpz

\bibliographystyle{gost2800}
\bibliography{rpz}

\begin{enumerate}[1)]
    \item A. Guttman. R-trees: A Dynamic Index Structure for Spatial Searching. Proceedings of ACM SIGMOD, pages 47-57, 1984
    \item N. Beckmann, H.P. Kriegel, R. Schneider and B. Seeger. The R*-tree: An Efficient and Robust Access Method for Points and Rectangles. Proceedings of ACM SIGMOD, pages 323-331, May 1990.
    \item N. Roussopoulos, S. Kelley and F. Vincent. Nearest Neighbor Queries. ACM SIGMOD, pages 71-79, 1995.
    \item Sahr, Kevin (2019). Central Place Indexing: Hierarchical Linear Indexing Systems for Mixed-Aperture Hexagonal Discrete Global Grid Systems. Cartographica: The International Journal for Geographic Information and Geovisualization, 54(1), 16–29. doi:10.3138/cart.54.1.2018-0022
    \item Федотовский П. В. и др. Сортировать или нет: экспериментальное сравнение R-Tree и B+-Tree в транзакционной системе для упорядоченной выдачи //Труды Института системного программирования РАН. – 2014. – Т. 26. – №. 4. – С. 73-90.
    \item  Jiajun Liu, Haoran Li, Yong Gao, Hao Yu, Dan Jiang. [IEEE 2014 22nd International Conference on Geoinformatics - Kaohsiung, Taiwan (2014.6.25-2014.6.27)] 2014 22nd International Conference on Geoinformatics - A geohash-based index for spatial data management in distributed memory. 2014
    \item Q. Liu, X. Tan, F. Huang, C. Peng, Y. Yao and M. Gao, "GB-Tree: An efficient LBS location data indexing method," 2014 The Third International Conference on Agro-Geoinformatics, Beijing, China, 2014, pp. 1-5, doi: 10.1109/Agro-Geoinformatics.2014.6910659. https://ieeexplore.ieee.org/document/6910659
    \item S. He, L. Chu and X. Li, "Spatial query processing for location based application on Hbase," 2017 IEEE 2nd International Conference on Big Data Analysis (ICBDA), Beijing, China, 2017, pp. 110-114, doi: 10.1109/ICBDA.2017.8078787. https://ieeexplore.ieee.org/abstract/document/8078787
    \item  Beckmann, N., Kriegel, H.-P., Schneider, R., Seeger, B.: The R*-tree: an efficient and robust access method for points and rectangles. In: Proceedings of the 1990 ACM SIGMOD International Conference on Management of Data, pp. 322–331 (1990). https://dl.acm.org/doi/abs/10.1145/93597.98741
    \item R. Bayer and E. McCreight, "Organization and maintenance of large ordered indices", Proceedings of the 1970 ACM SIGFIDET (Now SIGMOD) Workshop on Data Description Access and Control, 1970. https://ieeexplore.ieee.org/document/10322993
    \item Ganti, R.K., Srivatsa, M., Agrawal, D., Zerfos, P., Ortiz, J.: MP-trie: fast spatial queries on moving objects. In: Proceedings of the Industrial Track of the 17th International Middleware Conference, p. 1. ACM (2016) https://dl.acm.org/doi/abs/10.1145/3007646.3007653
    \item Sun, L.; Jin, B. Improving NoSQL Spatial-Query Processing with Server-Side In-Memory R*-Tree Indexes for Spatial Vector Data. Sustainability 2023, 15, 2442.
    \item Ибрагимов П.И. Сравнительный анализ алгоритмов геопоиска и пространственных индексов в высоконагруженных системах // 79-е дни науки студентов МИСиС: международные, межвузовские и институтские научно-технические конференции. Тезисы докладов. — М.: МИСиС, 2022.
    \item X. Guan, C. Bo, Z. Li and Y. Yu, "ST-hash: An efficient spatiotemporal index for massive trajectory data in a NoSQL database," 2017 25th International Conference on Geoinformatics, Buffalo, NY, USA, 2017, pp. 1-7, doi: 10.1109/GEOINFORMATICS.2017.8090927. keywords: {Spatiotemporal phenomena;Trajectory;Indexes;Erbium;Binary codes;ST-Hash;Spatiotemporal index;Spatiotemporal range query;Trajectory data;NoSQL},
    \item Membrey P., Plugge E., Hawkins T. The definitive guide to MongoDB: the
noSQL database for cloud and desktop computing. Apress, 2010
    \item Сидоров, И. Ю., Армяков, А. О., Байтин, А. А., & Гунин, А. В. (2015). Хранилище точечных геообъектов. In Минцевские чтения (pp. 234-247).
    \item Гулаков В.К. Многомерные структуры данных. Брянск, БГТУ, 2010, c. 387.
    \item Balkić Z., Šoštarić D., Horvat G. GeoHash and UUID identifier for multi-agent
systems. Agent and Multi-Agent Systems. Technologies and Applications. Berlin
Springer, Heidelberg, 2012, pp. 290–298
    \item Bohui, J. I. A. N. G., & Weifeng, Z. H. O. U. (2024). Comparative Analysis of GeoHash, Google S2 and Uber H3 as Global Geographic Grid Coding Methods. Geography & Geographic Information Science, 40(2).
    \item Comer, D. (1979). Ubiquitous B-tree. ACM Computing Surveys (CSUR), 11(2), 121-137.
    
\end{enumerate}

%%% Local Variables: 
%%% mode: latex
%%% TeX-master: "rpz"
%%% End: 

\chapter*{ПРИЛОЖЕНИЕ А\\\large \textbf{Основные фрагменты кода}}
\addcontentsline{toc}{chapter}{ПРИЛОЖЕНИЕ А Основные фрагменты кода}
    
\begin{lstlisting}[caption=points.go]
// Package geo
//
//	Point, Points method
package geo

import (
	"encoding/json"
	"geoindexing_comparison/backend/helpers"
	"github.com/paulmach/orb"
	"github.com/paulmach/orb/geojson"
	"golang.org/x/exp/slices"
	"math"
	"strings"

	"github.com/pkg/errors"

	"github.com/mmcloughlin/geohash"

	mapset "github.com/deckarep/golang-set/v2"
)

// Point represents a geographic coordinate.
type Point struct {
	ID string `json:"id"`

	Lat float64 `json:"lat"`
	Lon float64 `json:"lon"`
}
type Points []Point

func NewPoint(lat float64, lng float64) Point {
	return Point{
		ID:  helpers.ID(),
		Lat: lat,
		Lon: lng,
	}
}

func (r Point) Geohash(bits uint) uint64 {
	return geohash.EncodeIntWithPrecision(r.Lat, r.Lon, bits)
}

func (r Point) GeohashString(chars uint) string {
	return geohash.EncodeWithPrecision(r.Lat, r.Lon, chars)
}

func (r Point) GeoJSON() *geojson.Feature {
	feature := geojson.NewFeature(orb.Point{r.Lon, r.Lat})
	feature.Properties["ID"] = r.ID

	return feature
}

func (r Point) InsideBBox(bottomLeft Point, upperRight Point) bool {
	return bottomLeft.Lat < r.Lat && bottomLeft.Lon < r.Lon && r.Lat < upperRight.Lat && r.Lon < upperRight.Lon
}

func (r Point) AddLatitude(dvKM float64) Point {
	r.Lat = r.Lat + (dvKM/earthRadiusKm)*(180/math.Pi)
	return r
}

func (r Point) AddLongitude(dvKM float64) Point {
	r.Lon = r.Lon + (dvKM/earthRadiusKm)*(180/math.Pi)/math.Cos(r.Lat*math.Pi/180)
	return r
}

func (r *Points) GetRandomPoint() Point {
	return (*r)[helpers.RNG.IntN(len(*r))] //nolint: gosec // Allowed here
}

func (r *Points) FindCorners() (Point, Point) {
	bottomLeft, upperRight := (*r)[0], (*r)[0]
	for _, point := range *r {
		if point.Lat < bottomLeft.Lat && point.Lon < bottomLeft.Lon {
			bottomLeft = point
		}

		if point.Lat > upperRight.Lat && point.Lon > upperRight.Lon {
			upperRight = point
		}
	}

	return bottomLeft, upperRight
}

func (r *Points) Center() (float64, float64) {
	bottomLeft, upperRight := r.FindCorners()
	return (bottomLeft.Lat + upperRight.Lat) / 2.0, (bottomLeft.Lon + upperRight.Lon) / 2.0
}

func (r *Points) String() string {
	byteArray, err := json.Marshal(r)
	if err != nil {
		panic(errors.Wrap(err, "failed to marshal points"))
	}

	return string(byteArray)
}

func (r *Points) IDs() string {
	var ids []string
	for _, point := range *r {
		ids = append(ids, point.ID)
	}

	slices.Sort(ids)
	return strings.Join(ids, ",")
}

func (r *Points) Delete(pointID string) {
	for idx, point := range *r {
		if pointID == point.ID {
			*r = append((*r)[:idx], (*r)[idx+1:]...)
			return
		}
	}
}

func (r *Points) ToSet() mapset.Set[string] {
	result := mapset.NewSet[string]()
	for _, point := range *r {
		result.Add(point.ID)
	}

	return result
}

func (r *Points) GeoJSON() *geojson.FeatureCollection {
	featureCollection := geojson.NewFeatureCollection()
	for _, point := range *r {
		featureCollection.Append(point.GeoJSON())
	}

	return featureCollection
}

func (r *Points) SortByID() {
	slices.SortFunc(*r, func(a, b Point) int {
		return strings.Compare(a.ID, b.ID)
	})
}

func (r *Points) SortByDistance(origin Point) {
	slices.SortFunc(*r, func(a, b Point) int {
		if a.DistanceHaversine(origin) < b.DistanceHaversine(origin) {
			return -1
		}
		return 1
	})
}

func (r *Points) GetClosestViaSort(origin Point, n int) Points {
	if n > len(*r) {
		return *r
	}

	type dist struct {
		idx  int
		dist float64
	}

	knnMatrix := make([]dist, 0, len(*r))
	for idx, indexPoint := range *r {
		knnMatrix = append(knnMatrix, dist{idx: idx, dist: indexPoint.DistanceHaversine(origin)})
	}

	slices.SortFunc(knnMatrix, func(a, b dist) int {
		if a.dist < b.dist {
			return -1
		}

		return 0
	})

	result := make(Points, n)

	for idx := range n {
		result[idx] = (*r)[knnMatrix[idx].idx]
	}

	return result
}

func (r *Points) Equal(other Points) bool {
	return r.ToSet().Equal(other.ToSet())
}

func (r *Points) EqualMany(other []Points) bool {
	for _, otherPoint := range other {
		if !r.Equal(otherPoint) {
			return false
		}
	}

	return true
}

\end{lstlisting}

\begin{lstlisting}[caption=distance.go]
// Package geo
//
// Distance functions
package geo

import (
	"github.com/tidwall/geodesic"
	"math"
)

const earthRadiusKm = 6371 // radius of the earth in kilometers.

func distanceEuclidean(lat1, lon1, lat2, lon2 float64) float64 {
	return math.Sqrt(math.Pow(lat2-lat1, 2) + math.Pow(lon2-lon1, 2))
}

// degreesToRadians converts from degrees to radians.
func degreesToRadians(d float64) float64 {
	return d * math.Pi / 180
}

func distanceHaversine(lat1, lon1, lat2, lon2 float64) float64 {
	lat1 = degreesToRadians(lat1)
	lon1 = degreesToRadians(lon1)
	lat2 = degreesToRadians(lat2)
	lon2 = degreesToRadians(lon2)

	diffLat := lat2 - lat1
	diffLon := lon2 - lon1

	マギカ := math.Pow(math.Sin(diffLat/2), 2) + math.Cos(lat1)*math.Cos(lat2)*
		math.Pow(math.Sin(diffLon/2), 2)

	c := 2 * math.Atan2(math.Sqrt(マギカ), math.Sqrt(1-マギカ))

	return c * earthRadiusKm
}

func distanceGeodesic(lat1, lon1, lat2, lon2 float64) float64 {
	var dist float64
	geodesic.WGS84.Inverse(lat1, lon1, lat2, lon2, &dist, nil, nil)

	return dist
}

func (r Point) DistanceHaversine(other Point) float64 {
	return distanceHaversine(r.Lat, r.Lon, other.Lat, other.Lon)
}

func (r Point) DistanceGeodesic(other Point) float64 {
	return distanceGeodesic(r.Lat, r.Lon, other.Lat, other.Lon)
}

func (r Point) DistanceEuclidean(other Point) float64 {
	return distanceEuclidean(r.Lat, r.Lon, other.Lat, other.Lon)
}

\end{lstlisting}

\begin{lstlisting}[caption=bbox.go]
// Package geohash_utils
//
// Geohash utils, such as BBox
package geohash_utils

import (
	"github.com/mmcloughlin/geohash"
)

type BBox struct {
	bits       uint
	leftBottom uint64

	height int
	wight  int
}

func NewBBox(bottomLeftLat, bottomLeftLon, upperRightLat, upperRightLon float64, bits uint) BBox {
	var (
		wight  = 0
		height = 0

		bottomLeftHash  = geohash.EncodeIntWithPrecision(bottomLeftLat, bottomLeftLon, bits)
		upperLeftHash   = geohash.EncodeIntWithPrecision(upperRightLat, bottomLeftLon, bits)
		bottomRightHash = geohash.EncodeIntWithPrecision(bottomLeftLat, upperRightLon, bits)
	)

	for {
		if bottomLeftHash == upperLeftHash {
			break
		}
		upperLeftHash = NeighborIntWithPrecision(upperLeftHash, bits, geohash.South)
		height++
	}

	for {
		if bottomLeftHash == bottomRightHash {
			break
		}
		bottomRightHash = NeighborIntWithPrecision(bottomRightHash, bits, geohash.West)
		wight++
	}

	r := BBox{
		bits:       bits,
		leftBottom: bottomLeftHash,
		height:     height,
		wight:      wight,
	}

	return r
}

func collectPerimeter(hash uint64, bits uint, height, wight int) []uint64 {
	perimeter := make([]uint64, 0, height)

	for range height {
		perimeter = append(perimeter, hash)
		hash = NeighborIntWithPrecision(hash, bits, geohash.North)
	}

	for range wight {
		perimeter = append(perimeter, hash)
		hash = NeighborIntWithPrecision(hash, bits, geohash.East)
	}

	for range height {
		perimeter = append(perimeter, hash)
		hash = NeighborIntWithPrecision(hash, bits, geohash.South)
	}

	for range wight {
		perimeter = append(perimeter, hash)
		hash = NeighborIntWithPrecision(hash, bits, geohash.West)
	}

	return perimeter
}

// Perimeter
// Returns outer part of BBox
func (r *BBox) Perimeter() []uint64 {
	return collectPerimeter(r.leftBottom, r.bits, r.height, r.wight)
}

// Inner
// Returns inner part of BBox, points from Perimeter are not included
func (r *BBox) Inner() []uint64 {
	var (
		height    = r.height - 1
		wight     = r.wight - 1
		inner     = make([]uint64, 0, r.height)
		hash      = NeighborIntWithPrecision(r.leftBottom, r.bits, geohash.NorthEast)
		innerHash = hash
	)

	for range height {
		innerHash = hash
		for range wight {
			inner = append(inner, innerHash)
			innerHash = NeighborIntWithPrecision(innerHash, r.bits, geohash.East)
		}
		hash = NeighborIntWithPrecision(hash, r.bits, geohash.North)
	}

	return inner
}

\end{lstlisting}
\chapter{ПРИЛОЖЕНИЕ Б}
\section{Дополнительные результаты тестирований}
В данном приложение на рисунках 18-24 показаны результаты тестов по всем имеющимся задачам и всем имеющимся индексам. 

  \\
\begin{figure}[h]
    \centering
    \includegraphics[scale=0.17]{results_app_1.png}
    \caption{Задача BBox на квадрате с ребром 2км}
\end{figure}
  \\
    \\
\begin{figure}[h]
    \centering
    \includegraphics[scale=0.17]{results_app_2.png}
    \caption{Задача BBox на квадрате, в который входит не более 100 точек}
\end{figure}
  \\
    \\
\begin{figure}[h]
    \centering
    \includegraphics[scale=0.17]{results_app_3.png}
    \caption{Задача KNN на 100 точек}
\end{figure}
  \\
    \\
\begin{figure}[h]
    \centering
    \includegraphics[scale=0.17]{results_app_4.png}
    \caption{Задача KNN на 10 точек}
\end{figure}
  \\
    \\
\begin{figure}[h]
    \centering
    \includegraphics[scale=0.17]{results_app_5.png}
    \caption{Задача KNN на 1\% точек}
\end{figure}
  \\
    \\
\begin{figure}[h]
    \centering
    \includegraphics[scale=0.17]{results_app_7.png}
    \caption{Задача KNN на 25\% точек}
\end{figure}
  \\
    \\
\begin{figure}[h]
    \centering
    \includegraphics[scale=0.17]{results_app_6.png}
    \caption{Задача KNN на 90\% точек}
\end{figure}
  \\




\end{document}

%%% Local Variables:
%%% mode: latex
%%% TeX-master: t
%%% End:
