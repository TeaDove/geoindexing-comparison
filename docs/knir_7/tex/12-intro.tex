\Introduction

В настоящее время геоданные являются неотъемлемой частью многих высоконагруженных систем, таких как поисковые системы, социальные сети, картографические сервисы и другие. Однако, обработка и хранение большого объема геоданных может стать проблемой для разработчиков. Для решения данной проблемы были разработаны методы геопоиска и структуры геоиндексации, которые позволяют эффективно работать с геоданными в высоконагруженных системах.

Цель данной дипломной работы - провести сравнительный анализ применения методов геопоиска и структур геоиндексации в высоконагруженных системах. В работе будут рассмотрены основные принципы работы данных методов, их преимущества и недостатки. Также будет проведено сравнение производительности данных методов на различных наборах геоданных.

Под высоконагруженной системой подразумевается система, в которой присутствуте как больше количество чтений, то есть запросов типа SELECT, если использовать язык SQL, так и изменений данных: UPDATE, DELETE, INSERT. Такие системы в основном в моменте обслуживуют большой порядок пользователей и имеют большой RPS, например, 2000 запросов в секунду.

Основной проблемой работы с геоданными в высоконагруженных системах является неочевидность в выборе структур и методов хранения данных. Так, например, классический индекс B-Tree нельзя использовать при обращение к кортежам геоточек, потому что B-Tree умеет работать только с скалярными данными. При этом для работы с геоданным существует объемное количество индексов, например, R-Tree, KD-Tree, Quadtree, каждый из которых имеет свои плюсы и минусы. Помимо этого существует проблема сериализации и десериализации данных, которая заключается в том, что результаты анализа поиска, запросы на поиск и сами данных точек необходимо передавать по сети в как можно меньшем объеме и размере(в данном случае под объемом подразумевается количество передаваем данных, а размер - вес в байтах).

Результаты данной работы могут быть полезны для разработчиков высоконагруженных систем, которые работают с геоданными, а также для специалистов в области геоинформатики.
