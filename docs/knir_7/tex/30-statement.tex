\chapter{СОДЕРЖАТЕЛЬНАЯ ПОСТАНОВКА ЗАДАЧИ}
\label{cha:statement}

Целью данной работы является разработка программного обеспечения для анализа пространственных индексов в высоконагруженных системах. Данное ПО должно уметь анализировать как минимум предложенные пространственные индексы: R-tree, R*-tree, KD-tree, Geohash+trie, H3+trie. Анализ должен проводится в разрезе трех операций:
\begin{enumerate}
    \item  Insert(p point) - вставка нового объекта p в индекс
    \item  KNN(p point, k int) - поиск k ближайших объектов от точки p
    \item  RangeSearch(p point, r float) - поиск всех объектов, которых входят в круг, определяющийся центром p и радиусом r.
    \item  RectangleSearch(p point, r float) - поиск всех объектов, которых входят в прямоугольник с нижним левым углом в $(p_{lat} - r, p_{lng} - r)$ и верхним правым углом в  $(p_{lat} + r; p_{lng} + r)$.
\end{enumerate}

Важно отметить, что операция удаления точки не рассматривается из-за того, что в основном она не сильно отличается от операции вставки, а также из-за того, что в высоко нагруженных системах, в основном, применяется подход soft-detele(от англ. мягкое удаление), в котором данные не удаляются, а лишь помечаются в СУБД как неактивные. Также, при проведении анализа данных, в основном, работа происходит с read-only данными, то есть теми данными, которые не меняются.

Для решения поставленных целей необходимо решить следующие задачи:
\begin{enumerate}
    \item  Разработка программного обеспечения для тестирования пространственных индексов.
    \item  Тестирование и отладка разработанного программного обеспечения.
    \item  Анализ результатов работы программного обеспечения. Поиск наиболее подходящих структур под анализируемые задачи.
\end{enumerate}
