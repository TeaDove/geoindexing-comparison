\Conclusion

В ходе исследования был проведен сравнительный анализ применения алгоритмов геопоиска и пространственных индексов в высоконагруженных системах. Было выявлено, что все структуры имеют свои преимущества и недостатки, и выбор между ними зависит от конкретных задач и требований к системе.

Древесные методы R-tree, KD-tree позволяют быстро находить объекты на карте, но требует больших вычислительных ресурсов при обработке больших объемов данных. Geohash, Uber H3, S2 Geometry, в свою очередь, позволяют эффективно хранить и обрабатывать большие объемы геоданных, но могут быть менее точными в поиске объектов.

Таким образом, выбор между методами геопоиска должен основываться на конкретных требованиях к системе и ее возможностях. Важно учитывать как скорость поиска объектов, так и эффективность использования ресурсов системы.

При этом важно отметить, что в высоконагруженных системах данные методы могут комбинироваться для достижения наилучших результатов, например, использовать Geohash для поиска в местах, где не требуется точность, а R-tree для аналитики данных через холодные хранилища.

В работе удалось полностью достичь поставленных целей.

В следующем семестре будет решаться более прикладная проблема, а именно сравнительный анализ конкретных реализаций структур с целью выявления узких мест в реализациях. Потребуется реализовать все приведенные структуры и провести нагрузочное тестирование, чтобы можно было найти все крайние случаи работы указанных подходов. Для решения данной задачи будут применятся научная дисциплина Информатика, для реализации структур и проведения нагрузочного тестирования будет использоваться язык Golang.

За время выполнения работы мною было познано знание относительно формирования геохешей, а также устройство работы многомерных деревьев: R-tree, KD-tree и тд.
