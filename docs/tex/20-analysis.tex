\chapter{АНАЛИТИЧЕСКИЙ ОБЗОР ЛИТЕРАТУРЫ}
\label{cha:analysis}

В данной работе будут анализироваться только 2 задачи поиска по геоданным. Задачи можно сформулировать следующим образом:
Пусть есть множество точек, представляющих собой пару широта-долгота, отображающих некий объект на поверхности земли.

Надо:
\begin{enumerate}
    \item Найти все объекты Х, находящиеся на растояние У от объекта Z. Эту задачу можно переформулировать следующим образом: найти все объекты множества Х в окружности c радиусом Y и центром в Z.
    \item Найти X ближайщих объектов к объекту Z. Эта задача также известна как задача поиска ближайших соседей.
\end{enumerate}

\subsection{Расстояние между двумя точками}
Указанные выше задачи требуют обусловить понятие расстояния между двумя геоточками. Для этого введем понятие геоточки: это точка, представляющая собой кортеж A(x, y), где x и y - широта и долгота соответственно.
Расстоянием между двумя точками есть наименьшая примая на плоскости земного шара, при этом ее вычисление может быть разным в зависиомости от положеной задачи. \\
Поставим задачу:

Даны геоточки $A(x_1, y_1), B(x_2, y_2)$. Требуется найти растояние $d(A, B)$

\subsubsection{Евклидово расстояние}
В некоторых системах, например, расширение Postgis для СУБД Postgres для типа Geometry, используется евклидово расстоние:

$$
d(A, B)=\sqrt{(x_1-x_2)^2+(y_1-y_2)^2}
$$

Данный подход работает корректно только в части случаев, так, например, он выдаст правильный результат с точностью до 5 метров в случае поиска растояния в пределах Москвы, но, расстояние в Африке будет отличаться в 2-3 раза относительно более корректных подходов.
При это важно отметить, что данная дистанция отвечает аксиомам метрики, поэтому ее корректно использовать при сравнение расстояний между двумя точками, например, при решение задаче KNN(К-ближайших соседей).

\subsubsection{Расстояние гаверсинуса}
Расстояние гаверсинуса - это способ определения расстояния между двумя точками на поверхности Земли, учитывающий кривизну Земли. Оно используется в геопоиске для определения расстояния между заданной точкой и объектами в заданном радиусе.

Формула расстояния гаверсинуса выглядит следующим образом:

$$
d(A, B) = 2R \cdot \arcsin\left(\sqrt{\sin^2\left(\frac{x_2-x_1}{2}\right) + \cos(x_1) \cdot \cos(x_2) \cdot \sin^2\left(\frac{y_2-y_1}{2}\right)}\right)
$$

где d - расстояние между двумя точками в километрах, R - радиус Земли (приблизительно 6371 км), $x_1$ и $x_2$ - широты двух точек в радианах, $y_1$ и $y_2$ - долготы двух точек в радианах.

Эта формула позволяет определить расстояние между точками с точностью до нескольких метров, что делает ее очень полезной для геопоиска. Однако, она может быть достаточно ресурсоемкой при работе с большими объемами геоданных, что делает ее медленее по сравнению с Евклидовым расстоянием.

\subsubsection{Геодезическое расстояние}
Геодезическое расстояние - это расстояние между двумя точками на поверхности Земли, измеренное вдоль кратчайшей линии (геодезической линии) между этими точками. Геодезическая линия - это кривая на поверхности Земли, которая имеет наименьшую длину между двумя точками.

Геодезическое расстояние учитывает кривизну Земли и может отличаться от расстояния гаверсинуса, особенно на больших расстояниях и при использовании разных моделей формы Земли.

Для расчета геодезического расстояния используются различные методы, такие как метод Винсента, метод Гаусса-Крюгера и метод Хаверсина. Эти методы учитывают форму Земли и позволяют получить более точные результаты, чем простые формулы для расчета расстояния на плоскости.

Геодезическое расстояние широко используется в геопозиционировании, навигации, картографии и других областях, где требуется точное определение расстояния между двумя точками на поверхности Земли.

Формула вычисления геодезии:
$$
d(A, B) = R \cdot \arccos(\sin(x_1) \cdot \sin(x_1) + \cos(x_1) \cdot \cos(x_2) \cdot \cos(y_2 - y_1)
$$


\subsection{Структуры геоиндексации}
Как и с задачами поиска по \textit{плоскому} массиву скаляров, для поиска по геокординатам используются специальные структуры данных, которые позволяют оптимизировать операции поиска. \\
Указанные структуры можно разбить на 2 типа: древовидные и плоские. К первому виду относятся: R-tree, R*-tree, Quad-tree, K-d tree. К плоским относятся: Geohash, S2 и H3. \\
Расмотрим принцип работы древовидных структур на примере K-d tree.\\
\subsubsection{K-d tree}
K-d tree представляет собой дерево, позволяющее производить операции поиска в N-мерном пространстве. Расмотрим двухмерное K-d tree, которое также можно назвать K-2 tree.
Само K-D tree является обычным бинарным сбалонсированным деревом.
Алгоритм построенния K-d tree довольно прост:
\begin{enumerate}
    \item Происходит поиск центральной точки, то есть той точки, которая будет находится на сумарно меньшем растояние от всех точек. Данная точка ставиться в корень дерева k-d tree.
    \item Далее плоскость "разбивается" на 2 части по вертикальной оси
    \item "Слева" ищется "средняя точка", то есть та точка, которая по оси абсцис(широте) находится на сумарно меньшем расстояние до остальных. Данная точка записывает в левого ребенка корня дерева
    \item Аналогичная процедура повторяется справа.
    \item Аналогичная процедура повторяется для вновь созданых полотен, но уже с осью ординат(долготой)
    \item Данные процедуры повторяются со всеми точками.
\end{enumerate}
\subsection{Процесс поиска по K-d tree}
Процесс поиска K ближайших точек и всех точек в заданном радиусе очень поход на процесс поиска по бинарному дереву за тем исключением, что при сравнение по четным нодам идет по широте, а по нечетным - по долготе.

\subsubsection{Geohash}
Geohash(далее также геохеш), в отличие от древовидных структур преставляет собой обычный массив, поиск по которому можно реализовывать через тривиальные методы, например, через бинарное дерево поиска, красно-черные деревья или b-tree.
Самим геохешом называется строка, закодированная 32 разрядным алфавитом, перевод из 10-ой системы в указанный алфавит указан в таблице ниже.
\begin{center}
\begin{tabular}{ c|c c c c c c c c }
 Основание 10 & 0 & 1 & 2 & 3 & 4 & 5 & 6 & 7 \\
 Основание 32 & 0 & 1 & 2 & 3 & 4 & 5 & 6 & 7 \\
  \hline\hline
 Основание 10 & 8 & 9 & 10 & 11 & 12 & 13 & 14 & 15 \\
 Основание 32 & 8 & 9 & b & c & d & e & f & g \\
  \hline\hline
 Основание 10 & 16 & 17 & 18 & 19 & 20 & 21 & 22 & 23  \\
 Основание 32 & h & j & k & m & n & p & q & r \\
  \hline\hline
 Основание 10 & 24 & 25 & 26 & 27 & 28 & 29 & 30 & 31 \\
 Основание 32 & s & t & u & v & w & x & y & z \\
\end{tabular}
\end{center}
Данная строка однозначно декодируется в кортеж геокоординат с некой точностью, зависящей от количества символов в строке. Примеры:
\begin{enumerate}
    \item строка \texttt{ucft} дегодируется в прямоугольник площадью примерно 800 $ km^2 $ и центром в (55.81, 37.44).
    \item строка \texttt{ucft943} имеет тот же центр - (55.8236, 37.3116), но меньшую площадь $23409 m^2$
\end{enumerate}
Как можно наблюдать, чем выше количество символов, используемых в геохеше, чем выше точность получаемых координат, но при этом выше затрачиваемая память.
В данной работе не будет детально описываться процесс формарования геохеша за исключением базового принципа: \\
Сфера земли разбивается на практически равные прямоугольники, после чего каждому прямоугольнику присваивается номер в 32х ричной системе координат, номера присваиваются в порядке "змейкой", сначала самый левый-верхний, далее ниже от него, далее справа от самого левого-верхнего и тд. \\

\subsection{Сравнительный анализ}
Чтобы дать корректный ответ на вопрос о том, какую струтуру данных стоит использовать, введем следующие критерии сравнения:
- занимаемый объем памяти
- время предподготовки данных
- время поиска K ближайщих соседей
- время поиска по радиусу и точке
- время вставки точки
- время удаления точки

При этом для чистоты эксперемента стоит добавить следующие ограничения:
- Не будет учитываться время записи на диск. Предпологается, что все операции производятся при использование оперативной памяти.
- Требуется учитывать как теоретические показатели, то есть O-нотацию, так и практические результаты.


%%% Local Variables:
%%% mode: latex
%%% TeX-master: "rpz"
%%% End:
